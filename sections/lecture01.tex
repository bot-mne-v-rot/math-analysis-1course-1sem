\section{Лекция за 6.11.20}
\subsection{Компактность}
\begin{conj}
    $A, B_{\alpha}, \ {\alpha \in I}$ - множества. Множества $B_{\alpha}$ покрывают $A$
    означает, что $A \subset \bigcup\limits_{\alpha \in I} B_{\alpha}$ \\
    $B_{\alpha}$ - покрытие $A$
\end{conj}
\begin{conj}
    Открытое покрытие - покрытие открытыми множествами
\end{conj}
\begin{conj}
    $(X, \rho)$ - метрическое пространство. $K \subset X$.
    $K$ - компакт, если из любого покрытия $K$ открытыми множествами можно выбрать конечное подпокрытие 
    (Какое бы мы не взяли покрытие открытыми множествами, можно оставить конечное количество этих открытых множеств
    и они все равно будут покрывать $K$)
\end{conj}
\textbf{Примеры:} 
\begin{itemize}
    \item[] Любой отрезок на прямой
    \item[] Квадрат c границей и внутренностью в $\R^2$
\end{itemize}
\begin{theorem-non} \quad 

    \begin{enumerate}
        \item Если $K \subset Y \subset X$, то $K$ - компакт в $Y \Longleftrightarrow K$ - компакт в $X$ 
        \begin{proof} \quad
            
            ``$\Longrightarrow$'': Пусть $G_{\alpha}$ - открытые множества в $X$, покрывающие $K$
            $\Longrightarrow \mathcal{U}_{\alpha} = G_{\alpha}\cap Y$ - открытое множество в $Y$

            $K \subset Y$ и $K \subset \bigcup G_{\alpha} \Longrightarrow K \subset Y \cap \bigcup G_{\alpha} =
            \bigcup (Y \cap G_{\alpha}) = \bigcup \mathcal{U}_{\alpha}$, то есть $\mathcal{U}_{\alpha}$ - 
            покрытие $K$ открытыми множествами в $Y \Longrightarrow \exists \ \mathcal{U}_{\alpha_1}, \mathcal{U}_{\alpha_2}, \dots, \mathcal{U}_{\alpha_n}$, 
            такие, что $K \subset \bigcup\limits_{j = 1}^{n} \mathcal{U}_{\alpha_j} \subset \bigcup\limits_{j = 1}^{n} G_{\alpha_j} \Longrightarrow 
            G_{\alpha_1}, \dots, G_{\alpha_n}$ - конечное подпокрытие $K \Longrightarrow$ $K$ --- компакт в $X$

            ``$\Longleftarrow$'': Рассмотрим покрытие $K$ множествами $\mathcal{U}_{\alpha}$, открытыми в $Y$. Тогда $\mathcal{U}_{\alpha} = G_{\alpha} \cap Y$, 
            где $G_{\alpha}$ открыто в $X \Longrightarrow K \subset \bigcup \mathcal{U}_{\alpha} \subset \bigcup G_{\alpha} \Longrightarrow
            G_{\alpha}$ - октрытое множество в $X \Longrightarrow \exists \ G_{\alpha_1}, G_{\alpha_2}, \dots, G_{\alpha_n}$ покрывающие $K \Longrightarrow
            K \subset \bigcup\limits_{j=1}^{n} G_{\alpha_j} \Longrightarrow K = K \cap Y \subset 
            Y \cap \bigcup\limits_{j=1}^{n} G_{\alpha_j} = \bigcup\limits_{j=1}^{n} Y \cap G_{\alpha_j} = \bigcup\limits_{j=1}^{n} \mathcal{U}_{\alpha_j}
            \Longrightarrow \mathcal{U}_{\alpha_1}, \mathcal{U}_{\alpha_2}, \dots, \mathcal{U}_{\alpha_n}$ покрывают $K \Longrightarrow K$ - компакт в $Y$  
        \end{proof}
        \item $K$ - компакт $\Longrightarrow K$ - замкнут и ограничен
        \begin{proof} \quad

            Проверим, что дополнение $K$ открыто. Возьмем $a \notin K$ и покажем, что 
            $B_r(a) \subset X \setminus K$ для некоторого $r > 0$. Для $x \in K$ возьмем шарик $B_{\rho(x, a)/2}(x) =: \mathcal{U}_x$
            - открытое множество $\mathcal{U}_x \cap B_{\rho(x, a)/2}(a) \neq \varnothing \Longrightarrow K \subset \bigcup\limits_{x \in K} \mathcal{U}_x$.
            Так как $K$ - компакт, можно вырбрать $\mathcal{U}_{x_1}, \mathcal{U}_{x_2}, \dots, \mathcal{U}_{x_n}$, такие что $K \subset \bigcup\limits_{j = 1}^{n} \mathcal{U}_{x_j}
            \quad \mathcal{U}_{x_j} \cap B_{r_j}(a) = \varnothing, \; r_j = \rho(x_j, a)/2$. Если $r = min \ r_j$, то $\mathcal{U}_{x_j} \cap B_{r}(a) = \varnothing \Longrightarrow K \cap B_r(a) = \varnothing 
            \Longrightarrow B_r(a) \subset X \setminus K$ 

            Теперь проверим ограниченность. Возьмем $a \in X$. Тогда $K \subset \bigcup\limits_{n = 1}^{\infty} B_n(a)$.
            Если $x \in K$, то расстрояние между $x$ и $a$ будет конечным, а значит есть такой натуральный номер, который больше, 
            чем это расстояние $\Longrightarrow x \in B_n(a)$. Так как $K$ - компакт, можно выбрать конечное подпокрытие 
            $B_1(a), \dots, B_n(a) \qquad K \subset \bigcup\limits_{j = 1}^{n} B_j(a) = B_n(a) \Longrightarrow K$ содержится в каком то шаре $\Longrightarrow K$ - ограниченное множество 
        \end{proof}
        \item $K \subset \overset{\mathtt{\sim}}{K}$ и $\overset{\mathtt{\sim}}{K}$ - компакт. Если $K$ - замкнуто, то $K$ - компакт
        (Замкнутое подмножество компакта - компакт)
        \begin{proof} \quad

            Покроем $K$ открытыми множествами. Пусть $K \subset \bigcup \mathcal{U}_{\alpha}$ - открытые множества. 
            
            Тогда $\overset{\mathtt{\sim}}{K} \subset (\underbrace{(X \setminus K)}_{\text{Откр. мн-во}} \cup \bigcup \mathcal{U}_{\alpha})$.
            Так как $\overset{\mathtt{\sim}}{K}$ - компакт, можно выбрать конечное подпокрытие $X \setminus K: \mathcal{U}_{\alpha_1}, \mathcal{U}_{\alpha_2}, \dots, \mathcal{U}_{\alpha_n}$

            $\underset{\quad \rotatebox[origin=c]{120}{$ \subset $} K}{\overset{\mathtt{\sim}}{K}} \subset (X \setminus K) \cup \bigcup\limits_{j = 1}^{n} \mathcal{U}_{\alpha_j} \Longrightarrow K \subset \bigcup\limits_{j = 1}^{n} \mathcal{U}_{\alpha_j} \Longrightarrow K$ - компакт
            
        \end{proof}
    \end{enumerate}
\end{theorem-non}

\begin{theorem-non}
    $K_{\alpha}$ - семейство компактов, таких что пересечение любого их количества непусто.
    Тогда $\bigcap K_{\alpha} \neq \varnothing$ --- пересечение всех непусто.
    \begin{proof}
        Возьмём компакт $K_{\alpha_0}$. Предположим, что $\bigcap K_{\alpha} = \varnothing$. \\
        Тогда $K_{\alpha_0} \subset (X \setminus \bigcap\limits_{\alpha \neq \alpha_{0}} K_{\alpha}) = 
        \bigcup\limits_{\alpha \neq \alpha_{0}} \underbrace{(X \setminus K_{\alpha})}_\text{откр. мн-ва}$ --- покрытие $K_{\alpha_{0}}$ открытыми множествами \\
        $\Longrightarrow$ из него можно выделить конечное подпокрытие: $X \setminus K_{\alpha_{1}}, \dots, X \setminus K_{\alpha_{n}}$ \\
        $\Longrightarrow K_{\alpha_0} \subset \bigcap\limits_{j=1}^{n}(X \setminus K_{\alpha_{j}}) = X \setminus \bigcap\limits_{j = 1}^{n}(K_{\alpha_{j}}) \Longrightarrow
        \bigcap\limits_{j=0}^{n} K_{\alpha_{j}} = \varnothing$ противоречие. 
    \end{proof}
\end{theorem-non}
\follow \quad Если $K_1 \supset K_2 \supset K_3 \supset ...$ непустые компакты, то $\bigcap\limits_{n=1}^{\infty} K_n \neq \varnothing$ \\
Если у нас компакты вложены, то пересечение их конечного количества --- самый маленький из них, по условию он не пустой, значит пересечение непусто.

\begin{conj}
Секвенциальная компактность: из любой последовательности точек множества
$K$ можно выбрать подпоследовательность, которая сходится к какой-то точке из $K$.
\end{conj}
\textbf{Пример:} $[a,b]$ --- секвенциальный компакт. \\
$x_n$ --- ограниченная последовательность точек из $[a,b]$, по теореме
Больцано—Вейерштрасса мы из неё можем выбрать сходящуюся подпоследовательность, а
по предельному переходу в неравенстве предел этой подпоследовательности будет лежать в $[a,b]$.

\begin{theorem-non}
Всякое бесконечное подмножество компакта имеет предельную точку.
\begin{proof}
    От противного. Пусть $A \subset K$ бесконечное подмножество, такое что $A' = \varnothing$. \\
    $\Longrightarrow A$ --- замкнуто $\Longrightarrow$ $A$ --- компакт, причём ни одна из точек $A$ не является предельной. \\
    $\forall a \in A$ найдётся $B_{r_a}(a) \setminus a$, т.ч. $B_{r_a}(a) \setminus a \cap A = \varnothing$. \\
    Тогда $A \subset \bigcup\limits_{a\in A} B_{r_a}(a)$ --- бесокнечное объединение.
    Из него нельзя убрать ни одно множество (потому что каждая точка покрывается ровно одним множеством, каждое множество покрывает только свой центр), поэтому мы не можем выбрать конечное подпокрытие. Противоречие.
\end{proof}
\end{theorem-non}

\follow Компактность $\Longrightarrow$ секвенциальная компактность. (В метрическом пространстве)
\begin{proof}
    $K$ --- компакт, $x_n$ --- последовательость точек из $K$. \\
    $D = \{x_1,x_2,x_3,\dots \}$.
    \begin{itemize}
        \item[$\#D < +\infty:$] Какой-то $x_k$ повторяется бесконечное количество раз \\
        $\Longrightarrow$ можно выбрать стационарную подпоследовательность (её предел лежит в $K$).
        \item[$\#D = +\infty:$] $D$ имеет предельную точку $a$ (согласно теореме выше) $\Longrightarrow$ найдётся последовательность точек
        из $D$, которая к ней сходится. Выкинем из неё все повторы, переставим в правильном порядке и получим
        подпоследовательность $x_{n_1}, x_{n_2},\dots $ сходящуюся к $a$. $a \in K$, так как $K$ --- замкнуто.
    \end{itemize}
\end{proof}

\notice \quad

В Топологическом пространстве это неверно, но пример сложный, ты не поймёшь :( \\
Наоборот тоже неверно, там сложно

\begin{lemma}
    Лемма Лебега \\
    Пусть $K$ - секвенциально компактное множество
    и $K \subset \bigcup \mathcal{U_{\alpha}}$, где $\mathcal{U_{\alpha}}$ --- открытые мн-ва. \\
    Существует такое $\varepsilon > 0$, что $\forall x \in K$ шарик $B_{\varepsilon}(x)$ целиком покрывается каким-то $\mathcal{U_{\alpha}}$.
    \begin{proof}
        От противного. Пусть $\varepsilon = \frac{1}{n}$ не подходит. \\
         $B_{\frac{1}{n}}(x_n)$ не содержится целиком ни в одном из $\mathcal{U_{\alpha}}$.  \\
         Поскольку у нас секвенциальная компактность, выберем из $x_n$ сходящуюся подпоследовательность $x_{n_k} \rightarrow y \in K$.
         Тогда $y \in \mathcal{U_{\alpha}}$ --- открытое множество, тогда $\exists \varepsilon > 0,$ т.ч. $B_{\varepsilon}(y) \subset \mathcal{U_{\alpha_0}}$. \\
         Так как $\varrho(x_{n_k},y)\rightarrow 0$, начиная с некоторого номера $\varrho(x_{n_k},y) < \frac{\varepsilon}{2}$ и $\frac{1}{n_k} < \frac{\varepsilon}{2}$, \\
         тогда $B_{\frac{1}{n_k}}(x_{n_k}) \stackrel{?}{\subset} B_{\varepsilon}(y) \subset \mathcal{U_{\alpha_0}}$. \\
         $B_{\frac{1}{n_k}}(x_{n_k}) \subset B_{\frac{\varepsilon}{2}(x_{n_k})} \stackrel{?}{\subset} \mathcal{U_{\alpha_0}}$ \\
         Возьмём $z \in B_{\frac{\varepsilon}{2}(x_{n_k})} \Longrightarrow 
         \begin{cases}
             \rho(x_{n_k},y) < \frac{\varepsilon}{2} \\
             \rho(z,x_{n_k}) < \frac{\varepsilon}{2}
         \end{cases} \Longrightarrow$ 
        $\rho(z,y) \leqslant \rho (z, x_{n_k}) + \rho (x_{n_k},y) < \varepsilon \Longrightarrow z \in B_{\varepsilon}(y)$ \\
        Противоречие.
    \end{proof}
\end{lemma}

\begin{theorem-non}
    В метрическом пространстве компактность $=$ секвенциальная компактность.
    \begin{proof}
        "$\Longleftarrow:$" Возьмем покрытие $K$ открытыми множествами $\mathcal{U_{\alpha}}, \quad K \subset \bigcup \mathcal{U_{\alpha}}$ \\
        Возьмём $\varepsilon > 0$ из Леммы Лебега. Рассмотрим покрытие $K \subset \bigcup\limits_{x \in K} B_{\varepsilon}(x)$. \\
        Достаточно выбрать конечное подпокрытие из $\bigcup\limits_{x \in K}B_{\varepsilon}(x)$ \\
        Возьмём любой $x_1 \in K$, если $B_{\varepsilon}(x_1) \supset K$, то нашлось конечное подпокрытие. \\
        Иначе выберем $x_2 \in K \setminus B_{\varepsilon}(x_1)$, если $B_{\varepsilon}(x_1) \cup B_{\varepsilon}(x_2) \supset K$, то нашлось конечное подпокрытие \\
        Иначе $x_3 \in K \setminus (B_{\varepsilon}(x_1) \cup B_{\varepsilon}(x_2))$ и т.д.  \\
        Если в какой то момент процедура оборвалась, то нашлось конечное подпокрытие. \\
        Если он бесконечный, то $x_n \in K \setminus \bigcup\limits_{j = 1}^{n - 1} B_{\varepsilon}(x_j) \Longrightarrow x_n \notin B_{\varepsilon}(x_j)$ при $j < n$ 
        $\Longrightarrow \rho(x_n, x_j) \geqslant \varepsilon \quad \forall n > j \Longrightarrow$ пусть $x_{n_k}$ --- сходящаяся подпоследовательность, значит $x_{n_k}$ --- фундоментальна, а это не так
        $\Longrightarrow$ у неё нет сходящайся подпоследовательности. Это противоречит секвенциальной компактности.
    \end{proof}
\end{theorem-non}

%таймкод сука ёбаный вр от 1:34:25
