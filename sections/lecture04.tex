\section{Лекция за 27.11.20}
\subsection*{Еще немного непрерывности}
\begin{theorem-non}
    Непрерывный образ отрезка - отрезок 

    \begin{proof} \quad

        $f:[a, b] \longrightarrow \R$.
        %TODO: Делать уоминания теорем ссылочными на место, где мы их определеяем
        Тогда по теореме Вейерштрасса она достигает своих точных верхней и нижней граней.
        \begin{itemize}
            \item[] $m:= \min\limits_{x \in [a, b]}{f(x)}$ \qquad тогда $m = f(u)$
            \item[] $M:= \max\limits_{x \in [a, b]}{f(x)}$ \qquad тогда $M = f(v)$
        \end{itemize} для некоторых $a, v \in [a, b]$

        Посмотрим на $f$ на отрезке $[u, v]$. Тогда по теореме Больцано-Коши она принимает все значения между $f(u) = m$
        и $f(v) = M$. Тогда $[m, M] \subset f([a, b]) \subset [m, M] \Longrightarrow f([a, b]) = [m, M]$ 
    \end{proof}
\end{theorem-non}
\begin{theorem-non}
    Непрерывный образ промежутка - промежуток (возможно другого типа)

    \begin{proof} \quad

        $m:= \inf\limits_{x \in \langle a, b \rangle}{f(x)}, \; M:= \sup\limits_{x \in \langle a, b \rangle}{f(x)}$

        Случай $m = M$ тривиален. Пусть $m < y < M \Longrightarrow \begin{cases}
            \text{найдется } p \in \langle a, b \rangle, \text{ т.ч } f(p) > y \\
            \text{найдется } q \in \langle a, b \rangle, \text{ т.ч } f(q) < y
        \end{cases}$

        Посмотрим на $f$ на $[p, q]$. По т. Больцано-Коши она принимает на нем все промежуточные значения.
        В частности, и $y$.
        А так как $y$ мы брали произвольный, $y \in f(\langle a, b \rangle) \Longrightarrow \\
        (m, M) \subset f(\langle a, b \rangle)$. Также мы знаем, что $f(\langle a, b \rangle) \subset [a, b]$, что делает наше доказательство полным.
    \end{proof}
    \textbf{Пример:} $sin: (0, 2\pi) \longrightarrow \R$ \qquad Множество значений: $[-1, 1]$ 
\end{theorem-non}
\begin{theorem-non}
    Теорема о непрерывности обратной функции. $f : \langle a, b \rangle \longrightarrow \R$
    непрерывна, строго монотонна, $m:= \inf\limits_{x \in \langle a, b \rangle}{f(x)}, \; M:= \sup\limits_{x \in \langle a, b \rangle}{f(x)}$. 
    Тогда $f^{-1} : \langle m, M \rangle \longrightarrow \R$ непрерывна и строго монотонна.
    \begin{proof} \quad

        \quad \underline{Строгая монотонность}: Пусть $f$ строго возрастает (с убыванием ситуация аналогичная, различие в знаках).
        Докажем, что $x < y \Longleftrightarrow f(x) < f(y):$ \\ 
        $x < y \Longrightarrow f(x) < f(y), \ x = y \Longrightarrow f(x) = f(y), \ 
        x > y \Longrightarrow f(x) > f(y)$

        \quad \underline{Непрерывность}: Пусть $y \in \langle m, M \rangle. f^{-1}$ строго монотонна. 
        $A := \lim\limits_{u \rightarrow y-} f^{-1}(u) \leqslant B := f^{-1}(y) \leqslant \lim\limits_{u \rightarrow y+} f^{-1}(u) =: C$.
        Если оба равенства, то доказана непрерывность в точке $y$.
        
        Если один из знаком строгий, то $A < C$.

        $A = \lim\limits_{u \rightarrow y-} f^{-1}(u) = \sup\limits_{u < y} f^{-1}(u) \Longrightarrow f^{-1}(u) \leqslant A$ при $u < y$

        $C = \lim\limits_{u \rightarrow y+} f^{-1}(u) = \inf\limits_{u > y} f^{-1}(u) \Longrightarrow f^{-1}(u) \geqslant C$ при $u > y$

        Также мы знаем, что $f^{-1}(\langle m, M \rangle) = \langle a, b \rangle$, а это идет в разрез с вышесказанным. Значит мы пришли к противоречию
    \end{proof}  
\end{theorem-non}
\textbf{``Поясняющая картинка:''} \\
\begin{tikzpicture}
    \begin{axis}[
        xmin=-3,   xmax=3,
	    ymin=-3,   ymax=3,
        axis lines = left,
        xlabel = $x$,
        ylabel = {$f(x)$},
    ]
    %Below the red parabola is defined
    \addplot [
        domain=0:10,
        smooth,
        color=red,
    ]
    {x^2 - 3};
    \addlegendentry{$x^2 - 3, x > 0$}
    %Here the blue parabloa is defined
    \addplot [
        smooth,
        color=blue,
        ]
        {sqrt(x+3)};
    \addlegendentry{$\sqrt{x+3}$}
    \addplot [
        smooth,
        color=green,
    ]
    {x};
    \end{axis}
    \end{tikzpicture}

\subsection*{Элементарные функции}

\begin{theorem-non}
    При $0 < x < \frac{\pi}{2} \qquad \sin{x} < x < \tan{x}$
    \begin{proof} \quad 
        
        \begin{tikzpicture}[scale=3.0,cap=round,>=latex]
            % draw the coordinates
            \draw[->] (-1.5cm,0cm) -- (1.5cm,0cm) node[right,fill=white] {$x$};
            \draw[->] (0cm,-1.5cm) -- (0cm,1.5cm) node[above,fill=white] {$y$};
    
            % draw the unit circle
            \draw[thick] (0cm,0cm) circle(1cm);

            %draw angle
            \draw[thick,orange] ([shift=(10:1cm)]-0.8,-0.171) arc (0:45:0.18cm);

            %draw lines
            \draw[blue] (0cm,0cm) -- (45:1.41cm)
                 (45:1cm) -- (0:0.71cm)
                 (45:1.41cm) -- (0:1cm)
                 (0cm,0cm) -- (1cm,0cm);

            %draw dots
            \filldraw[black] (45:1cm) circle(0.4pt)
                     (45:1.41cm) circle(0.4pt)
                     (0cm:0cm) circle(0.4pt)
                     (0:1cm) circle(0.4pt)
                     (0:0.71cm) circle(0.4pt);

            %draw nodes
            \draw (-0.1cm,-0.1cm) node(z) {$0$}
                  (0.71cm,-0.1cm) node(h) {$H$}
                  (1.1cm,-0.1cm) node(b) {$B$}
                  (0.55cm,0.7cm) node(a) {$A$}
                  (0.85cm,1cm) node(c) {$C$}
                  (0.25cm,0.1cm) node(x) {$x$};
            
            \foreach \x/\xtext in {
                180/\pi}
                    \draw (\x:0.85cm) node[fill=white] {$\xtext$};

            \draw (-1.25cm,0cm) node[above=1pt] {$(-1,0)$}
                  (1.25cm,0cm)  node[above=1pt] {$(1,0)$}
                  (0cm,-1.25cm) node[fill=white] {$(0,-1)$}
                  (0cm,1.25cm)  node[fill=white] {$(0,1)$};
        \end{tikzpicture}

        $\sin{x} = AH \\
        \tg{x} = BC \\ 
        S_{\triangle AOB} = \frac{1}{2} OB \cdot AH = \frac{\sin{x}}{2}$

        $S_{\triangle BOC} = \frac{1}{2} OB \cdot CB = \frac{\tg{x}}{2}$
        
        $S_{\text{сектор } AOB} = \pi \cdot \frac{x}{2\pi} = \frac{x}{2}$

        $S_{\triangle AOB} = \frac{\sin{x}}{2} < S_{\text{сектор } AOB} = \frac{x}{2}< S_{\triangle BOC} = \frac{\tg{x}}{2}$
    \end{proof} 
    \follow \begin{enumerate}
        \item $\abs{\sin{x}} \leqslant \abs{x} \qquad x \in \R$ и если $x \neq 0$, то неравенство строгое
        \begin{proof}
            
        \end{proof}
    \end{enumerate}
\end{theorem-non}