\section{Неравенство между средними степенными \href{https://youtu.be/p9C57KDo1Yg?t=403}{\Walley}}
\begin{theorem-non}
    Неравенство между средними степенными

    $x_1, x_2, \dots, x_n > 0,\ p<q$. Тогда

    \[M_p:=\left(\frac{x_1^p+x_2^p+ \dots+x_n^p}{n}\right)^\frac{1}{p} \leqslant 
    \left(\frac{x_1^q+x_2^q+\dots+x_n^q}{n}\right)^\frac{1}q =: M_q\]
\end{theorem-non}

\begin{proof}
    \quad \\
    \begin{itemize} 
        \item Случай $q = 1,\ p>0$. 
        
        Докажем, что $M_p \leqslant M_1$.

        \[\frac{x_1^p+x_2^p+\cdots x_n^p}{n} \leqslant \left(\frac{x_1+x_2+\cdots+x_n}{n}\right)^p\]
    
        Это неравенство Йенсена для $f(x) = -x^p,\ p\in (0, 1),\ \lambda_1=\cdots=\lambda_n=\frac{1}{n}$
        \item Случай: $0 < p < q$. 
        
        Пусть $r:=\frac{p}{q}$. Тогда по предыдущему пункту $M_r \leqslant M_1$.
        
        Подставим $x_1^q,\dots, x_n^q$.
    
        \[\left(\frac{(x_1^q)^r+\cdots+(x_n^q)^r}{n}\right  )^\frac{1}{r} \leqslant
        \frac{x_1^q+x_n^q+\cdots+x_n^q}{n}\]
    
        \[\left(\frac{x_1^p+\cdots+x_n^p}{n}\right)^\frac{q}{p} \leqslant
        \frac{x_1^q+x_2^q+\cdots+x_n^q}{n}\]
    
        Теперь просто возведем в степень $\frac{1}{q}$ и получим результат.
        \item Случай: $p < q < 0$.
        
        По предыдущему пункту $M_{-q} \leqslant M_{-p}$. 
        
        Подставим $1/x_1, 1/x_2,\cdots, 1/x_n$

        \[ \left(\frac{(1/x_1)^{-q}+\cdots+(1/x_n)^{-q}}{n}\right)^{-\frac{1}{q}} \leqslant
        \left(\frac{(1/x_1)^{-p} + \cdots + (1/x_n)^{-p} }{n}\right)^{-\frac{1}{p}} \]
    
        Левая и правая части соответственно равны
    
        \[\left(\frac{x_1^q+\cdots+x_n^q}{n}\right)^{-\frac{1}{q}} \leqslant
        \left(\frac{x_1^p+x_2^p+\cdots+x_n^p}{n}\right)^{-\frac{1}{p}}\]
    
        Теперь возведем в степень $-1$ и получим искомое неравенство.
        \item Случай $p < 0 < q$. 
        
        Покажем, что  $M_p \leqslant $ среднее геометрическое $\leqslant M_q$

        \[M_p = \frac{x_1^p+\cdots + x_n^p}{n}) \geqslant \sqrt[n]{x_1^p x_2^p \dots x_n^p} 
        = (\sqrt[n]{x_1x_2\cdots x_n})^p\]
        \[ (\frac{x_1^p+\cdots + x_n^p}{n})^{\frac{1}{p}} \leqslant \sqrt[n]{x_1x_2\cdots x_n} \]
    
        Знак поменялся, так как $p < 0$.
    
        \[M_q = \left(\frac{x_1^q+\cdots + x_n^q}{n}\right)^\frac{1}{q} \geqslant
        \left(\sqrt[n]{x_1^qx_2^q\cdots x_n^q}\right)^\frac{1}{q} = \sqrt[n]{x_1x_2\cdots x_n}\]
    \end{itemize}
\end{proof}

\begin{theorem-non}
    $\lim\limits_{p \to 0} M_p = M_0$ (среднее геометрическое)
    
    $\lim\limits_{p \to \infty} M_p = \max(x_1, x_2,\cdots, x_n)$

    $\lim\limits_{p \to -\infty} M_p = \min(x_1, x_2, \cdots, x_n)$
\end{theorem-non}