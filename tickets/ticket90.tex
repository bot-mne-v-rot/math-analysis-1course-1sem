\section{Определение первообразной и неопределенного интеграла. Общий вид первообразной. Примеры функций не имеющих первообразную}

\begin{conj}
    $f: <a, b> \to \R$. $F:<a, b> \to \R$ - первообразная $f$, если

    $F'(x) = f(x)\ \forall x\in <a, b>$
\end{conj}

Не всякая функция имеет первообразную. Примеры
\begin{itemize}
    \item sign $x:= \begin{cases}
        1,\ x > 0\\
        0,\ x = 0\\
        -1,\ x < 0\\
    \end{cases}$
    Пусть sign $x = F'(x)$, рассмотрим отрезок $[-1, 1]$.

    $F'(1) = 1,\ F'(-1) = -1$, тогда по теореме Дарбу $F'$ принимает все промежуточные значения, а это не так.

    \item По этому примеру можно понять, что не подходит любая функция, у которой есть "скачок"
\end{itemize}

\begin{theorem}
    Непрерывная на промежутке функция имеет первообразную. Доказательства еще не было
\end{theorem}

\begin{theorem}
    $f, F: <a, b> \to \R,\ F$ - первообразная, $f$
    \begin{itemize}
        \item $F + c$ - первообразная $f$, где $C$ - любая константа
        \item $\Phi: <a, b> \to \R$ - первообразная $f \Rightarrow$ $\Phi = F+C$ для некоторой $C$
    \end{itemize}
\end{theorem}

\begin{proof}
    1) $(F(x)+C)' = F'(x) = f(x)$
    
    2) $(\Phi(x)-F(x))' = \Phi'(x) - F'(x) = 0 \Rightarrow \Phi - F$ - константа
\end{proof}

\begin{conj}
    Неопределенный интеграл - множество всех первообразных

    $\int f(x) dx$ или $\int f$
\end{conj}

Замечание. Для проверки равенства $\int f = F(x) + C$ достаточно доказать, что $F'(x) = f$