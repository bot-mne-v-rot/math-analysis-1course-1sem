\section{Теорема Больцано–Коши. Теорема о непрерывном образе отрезка}
\begin{theorem-non}
    Больцано-Коши (Если функция непрерывна на отрезке $[a, b]$, то она принимает все значения между $f(a)$ и $f(b)$)

    $f: [a, b] \longrightarrow \R$ непрерывна на $[a, b], C$ лежит между $f(a)$ и $f(b)$

    Тогда $\exists c\in (a, b)$, т.ч. $f(c) = C$
    \begin{proof}
        Пойдем от противного, пусть такой точки $c$ не существует, тогда: 

        $U = f^{-1}(-\infty, C) \\
        V = f^{-1}(C, +\infty)$

        $U, V$ - открытые множества. Они не пересекаются

        $[a, b] \subset (U \cup V)$ - противоречие со связностью множества $[a, b]$
    \end{proof}
\end{theorem-non}

\begin{theorem-non}
    Непрерывный образ отрезка - отрезок 

    \begin{proof} \quad

        $f:[a, b] \longrightarrow \R$.
        %TODO: Делать уоминания теорем ссылочными на место, где мы их определеяем
        Тогда по теореме Вейерштрасса она достигает своих точных верхней и нижней граней.
        \begin{itemize}
            \item[] $m:= \min\limits_{x \in [a, b]}{f(x)}$ \qquad тогда $m = f(u)$
            \item[] $M:= \max\limits_{x \in [a, b]}{f(x)}$ \qquad тогда $M = f(v)$
        \end{itemize} для некоторых $a, v \in [a, b]$

        Посмотрим на $f$ на отрезке $[u, v]$. Тогда по теореме Больцано-Коши она принимает все значения между $f(u) = m$
        и $f(v) = M$. Тогда $[m, M] \subset f([a, b]) \subset [m, M] \Longrightarrow f([a, b]) = [m, M]$ 
    \end{proof}
\end{theorem-non}