\section{Определения непрерывных отображений. Их равносильность. Непрерывность функций слева и справа. Теорема о стабилизации знака}

\subsection{Непрерывные отображения}
\begin{conj}
    Непрерывность функции в точке 
\end{conj}
$f : E \to Y, \, E \subset X$ и $a \in E$

$f$ непрерывна в точке $a$, если:
\begin{itemize}
    \item По Коши
    \[ \forall \, \varepsilon > 0 \;\; \exists \, \delta > 0 \;\; \forall \, x \in E : \, \rho_x(x, a) < \delta \Rightarrow \rho_y(f(x), f(a)) < \varepsilon \]
    \item На языке окрестностей
    \[ \forall \, \varepsilon > 0 \;\; \exists \, \delta > 0 : \, f(B_{\delta}(a) \cap E) \subset B_{\varepsilon}(f(a)) \]
    \item С пределами
    
    Если $a$ не является предельной точкой $E$, то условия не накладываются (в какой-то окрестности нет точек кроме самой $a$, следовательно функция автоматически непрерына).

    Если $a$ является предельной, то $\lim\limits_{x \to a} f(x) = f(a)$.
    \item По Гейне
    
    Если $x_n \in E$ и $\lim x_n = a$, то $\lim f(x_n) = f(a)$.
\end{itemize}
\begin{notice}
    \begin{enumerate}
        \item Эти определения равносильны.
        \item Есть непрерывность слева и справа (расписывается через односторонние пределы - TODO).
    \end{enumerate}
\end{notice}
\begin{theorem-non}
    Теорема о стабилизации знака
\end{theorem-non}
\begin{proof}
    TODO
\end{proof}