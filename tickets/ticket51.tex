\section{Определения непрерывных отображений. Их равносильность. Непрерывность функций слева и справа. Теорема о стабилизации знака}

\begin{conj}
    Непрерывность функции в точке
\end{conj}
$f : E \to Y, \, E \subset X$ и $a \in E$

$f$ непрерывна в точке $a$, если:
\begin{itemize}
    \item По Коши
    \[ \forall \, \varepsilon > 0 \;\; \exists \, \delta > 0 \;\; \forall \, x \in E : \, \rho_x(x, a) < \delta \Rightarrow \rho_y(f(x), f(a)) < \varepsilon \]
    \item На языке окрестностей
    \[ \forall \, \varepsilon > 0 \;\; \exists \, \delta > 0 : \, f(B_{\delta}(a) \cap E) \subset B_{\varepsilon}(f(a)) \]
    \item С пределами
    
    Если $a$ не является предельной точкой $E$, то условия не накладываются (в какой-то окрестности нет точек кроме самой $a$, следовательно функция автоматически непрерывна).

    Если $a$ является предельной, то $\lim\limits_{x \to a} f(x) = f(a)$.
    \item По Гейне
    
    Если $x_n \in E$ и $\lim x_n = a$, то $\lim f(x_n) = f(a)$.
\end{itemize}
\begin{notice}
    Эти определения равносильны (это очевидно, пруфов не будет).
\end{notice}

\begin{conj}
    Непрерывность слева

    Если $a$ не является предельной точкой $E$, то условия не накладываются (автоматически непрерывна). 
    В противном случае, $\lim\limits_{x \to a-} f(x)$ должен быть равен $f(a)$. 
\end{conj}
\begin{notice}
    Непрерывность справа расписывается аналогичным образом.
\end{notice}

\begin{theorem-non}
    Теорема о стабилизации знака

$f : E \to \R, \, E \subset X, \, a \in E$, $f$ непрерывна в точке $a$

Если $f(a) \neq 0$, то $\exists \, r > 0$ такой, что знак функции в $B_r(a) \cap E$ тот же, что и у $f(a)$.
\end{theorem-non}
\begin{proof}
    Пусть $f(a) > 0 \Rightarrow \lim\limits_{x \to a} f(x) = f(a) > 0$ (случай $f(a) < 0$ разбирается аналогично). Возьмем $\varepsilon = \frac{f(a)}{2}$. 
    Тогда $\exists \, \delta > 0$ такой, что если $x \in B_r(a) \cap E$, то $|f(x) - f(a)| < \varepsilon$,
    а все такие $f(x)$ положительны.
\end{proof}
\underline{Примеры}
\begin{enumerate}
    \item $f(x) = x$ \\ Непрерывная функция, в определении подходит $\delta = \varepsilon$.
    \item $f(x) = [x]$ \\ Не непрерывная функция.
\end{enumerate}