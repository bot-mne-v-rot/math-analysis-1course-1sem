\section{Локальные максимумы и минимумы. Необходимое условие экстремума}
\begin{conj}
    $f: E \to \mathbb{R}, \, a \in E$
    \begin{itemize}
        \item $a$ -- нестрогий локальный минимум, если
        \[ \exists \, \delta > 0 : \forall x \in E \cap (a - \delta, a + \delta) \quad f(x) \geqslant f(a) \]
        \item $a$ -- строгий локальный минимум, если
        \[ \exists \, \delta > 0 : \forall x \in E \cap (a - \delta, a + \delta), \, x \neq a \quad f(x) > f(a) \]
        \item $a$ -- нестрогий локальный максимум если
        \[ \exists \, \delta > 0 : \forall x \in E \cap (a - \delta, a + \delta) \quad f(x) \leqslant f(a) \]
        \item $a$ -- строгий локальный максимум, если
        \[ \exists \, \delta > 0 : \forall x \in E \cap (a - \delta, a + \delta), \, x \neq a \quad f(x) < f(a) \]
    \end{itemize}
    Локальный экстремум -- локальный минимум или максимум.
\end{conj}

\begin{theorem-non}
    Необходимое условие экстремума.

    $f: \langle a, b \rangle \to \mathbb{R}, \, x_0 \in (a, b)$

    Если $x_0$ -- точка экстремума и $f$ дифференцируема в точке $x_0$, то $f'(x_0) = 0$.
\end{theorem-non}

\begin{proof}
    Пусть $x_0$ -- точка локального максимума (случай локального минимума разбирается аналогично).

    Тогда $\exists \, \delta > 0$ такая, что если $x \in (x_0 - \delta, x_0 + \delta)$, то $f(x) \leqslant f(x_0)$.
    Следовательно, функция $f$, суженная на интервал $(x_0 - \delta, x_0 + \delta)$, принимает в точке $x_0$ свое максимальное значение. 
    Тогда по теореме Ферма $f'(x_0) = 0$. 
\end{proof}

\begin{notice}
    \begin{itemize}
        \item Обратное неверно. 
        Например, $f(x) = x^3$.
    
        $f'(0) = 0$, но 0 не является точкой экстремума.
        \item $f$ может быть не дифференцируема в точке экстремума.
        Например, $f(x) = |x|$.

        0 является точкой локального минимума, но $f$ не дифференцируема в 0.
    \end{itemize}
\end{notice}

Если $f'(x_0) = 0$ или $f'(x_0)$ не существует, то $x_0$ -- точка, подозрительная на экстремум.
