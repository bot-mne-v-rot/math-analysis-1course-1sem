% !TEX root = ../MatanColloc02.tex

\section{Эквивалентные нормы. Эквивалентность норм в $\R^d$ \href{https://youtu.be/E7inz4tp-6k?t=5471}{\Walley}}


\begin{conj}
    $X$ -- векторное пространство. $\norm{\cdot}$ и $\vertiii{\cdot}$
    -- нормы в $X$. Эти нормы \textbf{эквивалентны}, если $\exists
    c_1, c_2 : c_1 \norm{x} \leqslant \vertiii{x} \leqslant c_2 \norm{x} \,\,
    \forall x$.
\end{conj}

\notice 
\begin{enumerate}
    \item Нетрудно доказать, что это действительно является отношением
    эквивалентости.
    \item Сходимость по эквивалентным нормам равносильна. Т.е.
    $\lim x_n = a$ по норме $\norm{\cdot}$ $\Leftrightarrow$ 
    $\lim x_n = a$ по норме $\vertiii{\cdot}$.
    \begin{proof} $ $

        $\lim x_n = a$ по норме $\norm{\cdot} \Leftrightarrow
        \lim \norm{x_n - a} = 0$\\
        $\lim x_n = a$ по норме $\vertiii{\cdot} \Leftrightarrow
        \lim \vertiii{x_n - a} = 0$\\
        $0 \leqslant c_1 \norm{x_n - a} \leqslant \vertiii{x_n - a} 
        \leqslant c_2 \norm{x_n - a}$

        ``$\Longrightarrow$'':

        $\norm{x_n - a} \rightarrow 0$,
        $c_1 \norm{x_n - a} \leqslant \vertiii{x_n - a} 
        \leqslant c_2 \norm{x_n - a}$ $\xRightarrow{\text{2 мил.}}$
        $\vertiii{x_n - a} \rightarrow 0$.

        ``$\Longleftarrow$'':

        $\vertiii{x_n - a} \rightarrow 0$,
        $0 \leqslant c_1 \norm{x_n - a} \leqslant \vertiii{x_n - a}$ 
        $\xRightarrow{\text{2 мил.}}$
        $\norm{x_n - a} \rightarrow 0$.
    \end{proof}
\end{enumerate}

\begin{theorem-non}
    В $\R^d$ все нормы эквивалентны.
\end{theorem-non}
\begin{proof} $ $

    Пусть $\norm{\cdot}$ -- стандартная норма в $\R^d$, $P(\cdot)$
    -- другая норма в $\R^d$.

    $x := \sum_{k = 1}^d x_k e_k$, где $e_k = (0, ..., 0, 
    \overset{k\text{-e место}}{1}, 0, ..., 0)$.

    $P(x - y) = P(\sum_{k = 1}^d (x_k - y_k) e_k) \leqslant
    \sum_{k = 1}^d P((x_k - y_k) e_k) = \sum_{k = 1}^d \abs{x_k - y_k} 
    P(e_k) \underset{\text{К-Б}}{\leqslant} \sqrt{\sum_{k = 1}^d 
    (x_k - y_k)^2} \cdot \sqrt{\sum_{k = 1}^d P(e_k)^2} = \norm{x - y}
    \underbrace{\sqrt{\sum_{k = 1}^d P(e_k)^2}}_{M} = M \norm{x - y}$.

    $0 \leqslant P(x - y) \leqslant M \norm{x - y} \Rightarrow P$ -- 
    непрерывная функция, $P(x) \leqslant M \norm{x}$. Доказали неравенство
    с одной стороны.

    Рассмотрим $K$ -- единичная сфера в $\R^d$ -- компакт $\Rightarrow$
    найдётся $a \in K$, $a \neq \overline{0}$, т.ч. $0 < P(a) \leqslant P(x) 
    \,\, \forall x \in K$ по теореме Вейерштрасса.

    Возьмём $y\in \R^d, y \neq \overline{0} \Rightarrow y = \norm{y}
    \cdot \frac{y}{\norm{y}} \Rightarrow P(y) = \norm{y} \cdot 
    P(\frac{y}{\norm{y}}) \geqslant \norm{y} \cdot P(a)$, т.к. $\norm{a}
    \leqslant 1$. Доказали неравенство с другой стороны.

    $P(a) \norm{y} \leqslant P(y) \leqslant M \norm{y}$.
\end{proof}
