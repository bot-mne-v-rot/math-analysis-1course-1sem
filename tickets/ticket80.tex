\section{Формулы Тейлора для $e^x, \sin{x}, \cos{x}, \ln{(1+x)}$ и $(1+x)^p$. Разложения $\sin{x}, \cos{x}, e^x$ в ряд}

\subsection*{Формулы Тейлора для элементарных функций (все при $x \rightarrow 0$, остатки в форме Пеано)}

\begin{center}

    $e^x = 1 + x + \frac{x^2}{2!} + \frac{x^3}{3!} + \dots + \frac{x^n}{n!} + o(x^n)$

    \begin{flushleft}
    \textbf{Пояснение:}

    Производная экспоненты --- сама экспонента, в нуле это всегда единица.
    \end{flushleft}
    $\quad$ \\
    $\cos{x} = 1 - \frac{x^2}{2!} + \frac{x^4}{4!} - \frac{x^6}{6!} + \dots + \frac{(-1)^n}{(2n)!} + o(x^{2n + 1})$

    \begin{flushleft}
    \textbf{Пояснение:}

    $cos^{(k)}(0) = cos(0 + \frac{\pi \cdot k}{2})$

    Коэффициенты:
    $\begin{cases}
        0, k$ --- нечётно $ \\
        1, k \; mod \; 4 = 0 \\
        -1, k \; mod \; 4 = 2
    \end{cases}$
    \end{flushleft}
    $\quad$ \\
    $sin x = x - \frac{x^3}{3!} + \frac{x^5}{5!} - \frac{x^7}{7!} - \dots + \frac{(-1)^n \cdot x^{2n+1}}{(2n+1)!} + o(x^{2n+2})$

    \begin{flushleft}
    \textbf{Пояснение:}

    $sin^{(k)}(0) = sin(0 + \frac{\pi \cdot k}{2})$

    Коэффициенты:
    $\begin{cases}
        0, k$ --- чётно $\\
        1, k \; mod \; 4 = 1 \\
        -1, k \; mod \; 4 = -1
    \end{cases}$
    \end{flushleft}
    $\quad$ \\
    $ln(1 + x) = x - \frac{x^2}{2} + \frac{x^3}{3} - \frac{x^4}{4} + \frac{x^5}{5} - \dots + \frac{(-1)^{n-1} \cdot x^n}{n} + o(x^n)$
    
    \begin{flushleft}
    \textbf{Пояснение:}

    $ln(1 + x)^{(k)} = \frac{(-1)^{k-1}}{(1 + x)^k} \cdot (k - 1)!$
    
    в нуле $(-1)^{k-1} \cdot (k - 1)!$
    \end{flushleft}
    $\quad$ \\
    $(1 + x)^p = 1 + px + \frac{p(p-1)}{2!} \cdot x^2 + \frac{p(p-1)(p-2)}{3!} \cdot x^3 + \dots + \frac{p(p-1) \dots (p - n + 1)}{n!} \cdot x^n + o(x^n)$ 

    \begin{flushleft}
    \textbf{Пояснение:}

    $((1 + x)^p)^{(k)} = p(p-1) \dots (p - k + 1)(1 + x)^{p-k}$

    в нуле $p(p-1) \dots (p - k + 1)$
    \end{flushleft}
\end{center}

\subsection*{Ряды Тейлора}

\begin{gather*}
    e^x = \sum_{n = 0}^{\inf} \frac{x^n}{n!}  \\
    sin x = \sum_{n = 0}^{\inf} (-1)^n \frac{x^{2n+1}}{(2n+1)!} \\
    cos x = \sum_{n = 0}^{\inf} (-1)^n \frac{x^{2n}}{(2n)!} \\
\end{gather*}

\begin{proof}
    $\quad$ \\
    \begin{enumerate}
    \item 
    По следствию $2$:

    $f(x) = sin x, \quad f^{(n)}(x) = \pm sin x$ или $\pm cos x \Longrightarrow | f^{(n)}(x)| \leqslant 1 \quad \forall x$

    то же самое для $f(x) = cos x$

    \item 
    $f(x) = e^x, \quad f^{(n)}(x) = e^x$

    Рассмотрим $[0, x]$ (НУО $x > 0$): на нём $f^{(n)}(t) = e^t \leqslant max \{ 1, e^x \}$

    по следствию $2$ на этом отрезке всё ок.

    \end{enumerate}
\end{proof}

