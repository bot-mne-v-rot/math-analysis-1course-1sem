\section{Секвенциальная компактность. Компактность и предельные точки. Секвенциальная компактность компакта \href{https://youtu.be/IwHCWoW4oes?t=2678}{\Walley}}

\begin{conj}
    Секвенциальная компактность: из любой последовательности точек множества
    $K$ можно выбрать подпоследовательность, которая сходится к какой-то точке из $K$.
\end{conj}

\textbf{Пример:}
    $[a,b]$ --- секвенциальный компакт. \\
    $x_n$ --- ограниченная последовательность точек из $[a,b]$, по теореме
    Больцано—Вейерштрасса мы из неё можем выбрать сходящуюся подпоследовательность, а
    по предельному переходу в неравенстве предел этой подпоследовательности будет лежать в $[a,b]$.
    
\begin{theorem-non}
    Всякое бесконечное подмножество компакта имеет предельную точку.

    \begin{proof}
        От противного. Пусть $A \subset K$ бесконечное подмножество, такое что $A' = \varnothing$. \\
        $\Longrightarrow A$ --- замкнуто $\Longrightarrow$ $A$ --- компакт, причём ни одна из точек $A$ не является предельной. \\
        $\forall a \in A$ найдётся $\overset{\circ}{B}_{r_a}(a)$, т.ч. $\overset{\circ}{B}_{r_a}(a) \cap A = \varnothing$. \\
        Тогда $A \subset \bigcup\limits_{a\in A} B_{r_a}(a)$ --- бесконечное объединение.
        Из него нельзя убрать ни одно множество (потому что каждая точка покрывается ровно одним множеством, каждое множество покрывает только свой центр), поэтому мы не можем выбрать конечное подпокрытие. Противоречие.
    \end{proof}

\end{theorem-non}
    
\follow \; Компактность $\Longrightarrow$ секвенциальная компактность. (В метрическом пространстве)

\begin{proof}
    $K$ --- компакт, $x_n$ --- последовательость точек из $K$. \\
    $D = \{x_1,x_2,x_3,\dots \}$.

    \begin{itemize}
        \item[] $\#D < +\infty:$ Какой-то $x_k$ повторяется бесконечное количество раз \\
        $\Longrightarrow$ можно выбрать стационарную подпоследовательность (её предел лежит в $K$).
        \item[] $\#D = +\infty: D$ имеет предельную точку $a$ (согласно теореме выше) $\Longrightarrow$ найдётся последовательность точек
        из $D$, которая к ней сходится. Выкинем из неё все повторы, переставим в правильном порядке и получим
        подпоследовательность $x_{n_1}, x_{n_2},\dots $ сходящуюся к $a$. $a \in K$, так как $K$ --- замкнуто.
    \end{itemize}

\end{proof}
    
\notice \quad В Топологическом пространстве это неверно, но пример сложный, ты не поймёшь :( \\
Наоборот тоже неверно, там сложно