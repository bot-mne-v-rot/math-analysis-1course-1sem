% !TEX root = ../MatanColloc02.tex

\section{Формулы Тейлора с остатком в форме Пеано (с леммой). Единственность многочлена Тейлора \href{https://youtu.be/au9-34CerJM?t=4684}{\Walley}}

\begin{lemma}
    
    $g$ дифференцируема $n$ раз в точке $x_0$ и $g(x_0) = g'(x_0) = \dots = g^{(n)}(x_0) = 0$

    Тогда $g(x) = o((x - x_0)^n)$

    \begin{proof}
        
        Надо доказать, что $\lim\limits_{x \to x_o} \frac{g(x)}{(x-x_0)^n} = 0$,
        будем раскрывать по Лопиталю.
        \begin{gather*}
            \lim\limits_{x \to x_o} \frac{g(x)}{(x-x_0)^n} = \lim\limits_{x \to x_o} \frac{g'(x)}{n(x-x_0)^{n-1}} \\ 
            = \lim\limits_{x \to x_o} \frac{g''(x)}{n(n-1)(x-x_0)^{n-2}} = \dots = \lim\limits_{x \to x_o} \frac{g^{(n-1)}(x)}{n!(x-x_0)}
        \end{gather*}

        Знаем, что $g^{(n-1)}(x) = g^{(n-1)}(x_0) + g^{(n)}(x_0) \cdot (x - x_0) + o(x - x_0)$, вспоминаем,
        что у нас производные нули, и понимаем, что это всё равно $o(x - x_0)$, поделив это в пределе на $n!(x-x_0)$ мы получим $0$.

    \end{proof}

\end{lemma}


\begin{theorem-non}
    Формула Тейлора с остатком в форме Пеано

    $f$ дифференцируема $n$ раз в точке $x_0$

    Тогда 
    \begin{gather*}
        f(x) = T_{n, x_0} f(x) + o((x - x_0)^n) = \sum_{k=0}^{n} \frac{f^{(k)}(x_0)}{k!} \cdot (x - x_0)^k + o((x - x_0)^n)
    \end{gather*}

    \begin{proof}

            \begin{gather*}
                g(x) := f(x) - T_{n, x_0} f(x) \\
                g^{(m)}(x_0) = f^{(m)}(x_0) - (\sum_{k = 0}^{n} \frac{f^{(k)}(x_0)}{k!} \cdot (x - x_0)^k)^{(m)} |_{x = x_0} = \\
                = f^{(m)}(x_0) - \frac{f^{(m)}(x_0)}{m!} \cdot m! = 0
            \end{gather*}

            Всё это работает при $m \leqslant n$.

            Тогда по лемме $g(x) = o((x - x_0)^n)$
    \end{proof}
\end{theorem-non}

\begin{follow}
    Если $f$ дифференцируема $n$ раз в точке $x_0$ и $P$ --- многочлен степени $\leqslant n$, т.ч.
    $f(x) = P(x) + o((x - x_0)^n)$ при $x \rightarrow x_0$, то $P(x) = T_{n,x_0} f(x)$

    \begin{proof}
        
        \begin{gather*}
            f(x) = P(x) + o((x - x_0)^n) = T_{n,x_0} f(x) + o((x - x_0)^n)  \\
            \Longrightarrow Q(x) := T_{n, x_0} f(x) - P(x) = o((x - x_0)^n) \\
            deg Q \leqslant n
        \end{gather*}

        $Q(x) = \sum_{k = 0}^{n} c_k \cdot (x - x_0)^k $
        
        Пусть $c_m \neq 0$, где $m$ --- наименьший такой индекс.

        $Q(x) = c_m(x - x_0)^m + o((x - x_0)^m) = o((x - x_0)^n) \Longrightarrow c_m + o(1) = o((x - x_0)^{n - m})$

        Противоречие.

    \end{proof}
\end{follow}