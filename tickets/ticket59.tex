\section{Теорема о непрерывном образе промежутка. Непрерывность обратной функции}
\begin{theorem-non}
    Непрерывный образ промежутка - промежуток (возможно другого типа)

    \begin{proof} \quad

        $m:= \inf\limits_{x \in \langle a, b \rangle}{f(x)}, \; M:= \sup\limits_{x \in \langle a, b \rangle}{f(x)}$

        Случай $m = M$ тривиален. Пусть $m < y < M \Longrightarrow \begin{cases}
            \text{найдется } p \in \langle a, b \rangle, \text{ т.ч } f(p) > y \\
            \text{найдется } q \in \langle a, b \rangle, \text{ т.ч } f(q) < y
        \end{cases}$

        Посмотрим на $f$ на $[p, q]$. По т. Больцано-Коши она принимает на нем все промежуточные значения.
        В частности, и $y$.
        А так как $y$ мы брали произвольный, $y \in f(\langle a, b \rangle) \Longrightarrow \\
        (m, M) \subset f(\langle a, b \rangle)$. Также мы знаем, что $f(\langle a, b \rangle) \subset f([a, b])$, что делает наше доказательство полным.
    \end{proof}
    \textbf{Пример:} $sin: (0, 2\pi) \longrightarrow \R$ \qquad Множество значений: $[-1, 1]$ 
\end{theorem-non}
\begin{theorem-non}
    Теорема о непрерывности обратной функции. 
    
    $f : \langle a, b \rangle \longrightarrow \R$
    непрерывна, строго монотонна, $m:= \inf\limits_{x \in \langle a, b \rangle}{f(x)}, \; M:= \sup\limits_{x \in \langle a, b \rangle}{f(x)}$. 
    Тогда $f^{-1} : \langle m, M \rangle \longrightarrow \R$ непрерывна и строго монотонна.
    \begin{proof} \quad

        \quad \underline{Строгая монотонность}: Пусть $f$ строго возрастает (с убыванием ситуация аналогичная, различие в знаках).
        Докажем, что $x < y \Longleftrightarrow f(x) < f(y):$ \\ 
        $x < y \Longrightarrow f(x) < f(y), \ x = y \Longrightarrow f(x) = f(y), \ 
        x > y \Longrightarrow f(x) > f(y)$

        \quad \underline{Непрерывность}: Пусть $y \in \langle m, M \rangle. f^{-1}$ строго монотонна. 
        $A := \lim\limits_{u \rightarrow y-} f^{-1}(u) \leqslant B := f^{-1}(y) \leqslant \lim\limits_{u \rightarrow y+} f^{-1}(u) =: C$.
        Если оба равенства, то доказана непрерывность в точке $y$.
        
        Если один из знаком строгий, то $A < C$.

        $A = \lim\limits_{u \rightarrow y-} f^{-1}(u) = \sup\limits_{u < y} f^{-1}(u) \Longrightarrow f^{-1}(u) \leqslant A$ при $u < y$

        $C = \lim\limits_{u \rightarrow y+} f^{-1}(u) = \inf\limits_{u > y} f^{-1}(u) \Longrightarrow f^{-1}(u) \geqslant C$ при $u > y$

        Также мы знаем, что $f^{-1}(\langle m, M \rangle) = \langle a, b \rangle$, а это идет в разрез с вышесказанным. Значит мы пришли к противоречию
    \end{proof}  
\end{theorem-non}
\textbf{``Поясняющая картинка:''} \\
\begin{tikzpicture}
    \begin{axis}[
        xmin=-3,   xmax=3,
	    ymin=-3,   ymax=3,
        axis lines = left,
        xlabel = $x$,
        ylabel = {$f(x)$},
    ]
    %Below the red parabola is defined
    \addplot [
        domain=0:10,
        smooth,
        color=red,
    ]
    {x^2 - 3};
    \addlegendentry{$x^2 - 3, x > 0$}
    %Here the blue parabloa is defined
    \addplot [
        smooth,
        color=blue,
        ]
        {sqrt(x+3)};
    \addlegendentry{$\sqrt{x+3}$}
    \addplot [
        smooth,
        color=green,
    ]
    {x};
    \end{axis}
\end{tikzpicture}
