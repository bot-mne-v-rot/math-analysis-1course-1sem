\section{Теорема о дифференцируемости композиции}
\begin{theorem-non} 
    Дифференцируемость композиции

    Пусть 
    \begin{itemize}
        \item $f: \langle a, b \rangle \to \mathbb{R}, \; g: \langle c, d \rangle \to \mathbb{R}$
        \item $x_0 \in \langle c, d \rangle, \; g$ дифференцируема в точке $x_0$, $f$ дифференцируема в точке $g(x_0)$.
    \end{itemize}

    Тогда $f \cdot g$ дифференцируема в точке $x_0$ и $(f \cdot g)'(x_0) = f'(g(x_0)) * g'(x_0)$
\end{theorem-non}
\begin{proof} \quad \\
    $y_0 = g(x_0)$

    $f(y) - f(y_0) = \varphi(y)(y - y_0)$ для некоторой $\varphi$, непрерывной в точке $y_0$

    $g(x) - g(x_0) = \psi(x)(x - x_0)$ для некоторой $\psi$, непрерывной в точке $x_0$

    $\Rightarrow f(g(x)) - f(g(x_0)) = \varphi(g(x)) * (g(x) - g(x_0)) = \varphi(g(x))\psi(x) * (x - x_0)$

    Надо понять, что функция $\varphi(g(x))\psi(x)$ непрерывна в точке $x_0$.

    Это так, потому что $\psi(x)$ непрерывна по определению, а $\varphi(g(x))$ непрерывна как композиция непрерывных.

    Кроме того, $(f \cdot g)'(x_0) = \varphi(g(x_0))\psi(x_0) = \varphi(y_0)\psi(x_0) = f'(y_0)g'(x_0) = f'(g(x_0))g'(x_0)$
\end{proof}