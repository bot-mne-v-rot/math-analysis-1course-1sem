\section{Три определения предела отображения в точке. Равносильность определений. Единственность предела}
\begin{conj} 
    Предел отображения 
\end{conj}
$(X, \rho_x), \; (Y, \rho_y)$ -- метрические пространства 

$E \subset X, \; a$ -- предельная точка $E$

$f : E \to Y$
\[ \lim_{x \to a} f(x) = b \]
\begin{itemize}
    \item По Коши
    \[ \forall \, \varepsilon > 0 \;\; \exists \, \delta > 0 \;\; \forall x \in E \;\; 0 < \rho_x(x, a) < \delta \Rightarrow \rho_y(f(x), b) < \varepsilon \]
    \item На языке окрестностей
    \[ \forall \, B_{\varepsilon}(b) \;\; \exists \; \overset{\circ}{B}_{\delta}(a) : f(\overset{\circ}{B}_{\delta}(a) \cap E) \subset B_{\varepsilon}(b) \]
    \item По Гейне
    
    Для любой последовательности $x_n \in E$ и $x_n \neq a$ если $\lim x_n = a$, то $\lim f(x_n) = b$
\end{itemize}

\begin{notice}
    \begin{enumerate}
        \item В определении не участвует значение в точке $a$.
        \item Определение локальное.
    \end{enumerate}
\end{notice}

\begin{conj} 
    Предел функции 
\end{conj}
$f : E \to \R$

$E \subset \R, \; a$ -- предельная точка $E$

$b \in \R$
\[ \forall \, \varepsilon > 0 \;\; \exists \, \delta > 0 \;\; \forall x \in E \;\; 0 < |x - a| < \delta \Rightarrow |f(x) - b| < \varepsilon \]
\begin{notice}
    Это определение по Коши, оределеления на языке окрестностей и по Гейне аналогичны определению предела отображения.
\end{notice}

\begin{theorem-non}
    Определения по Коши и по Гейне равносильны.
\end{theorem-non}
\begin{proof} \quad \\
    Коши $\Rightarrow$ Гейне: 
    
    Берем последовательность из определения по Гейне:
    \[ x_n \in E, \; x_n \neq a \;\; \lim x_n = a \]
    Зафиксируем $\varepsilon > 0$ из определения по Коши, по нему найдем $\delta$, по $\delta$ найдем $N$ из определения по Гейне:
    \begin{gather*}
        \exists \, N \;\; \forall \, n \geqslant N \;\; \rho_x(x_n, a) < \delta \Rightarrow \rho_y(f(x_n), b) < \varepsilon \\
        \Rightarrow \lim f(x_n) = b
    \end{gather*}
    Гейне $\Rightarrow$ Коши:

    Зафиксируем $\varepsilon > 0$ и предположим, что для него не найдется $\delta > 0$.

    Не подходит $\delta = 1$, т.е. найдется $x_1 \in E, \; x_1 \neq a$ такой, что $\rho_x(x_1, a) < 1$ и $\rho_y(f(x_1), b) \geqslant e$. \\
    Не подходит $\delta = \frac{1}{2}$, т.е. найдется $x_2 \in E, \; x_2 \neq a$ такой, что $\rho_x(x_2, a) < \frac{1}{2}$ и $\rho_y(f(x_2), b) \geqslant e$. \\
    И так далее. В общем случае: \\
    Не подходит $\delta = \frac{1}{n}$, т.е. найдется $x_n \in E, \; x_n \neq a$ такой, что $\rho_x(x_n, a) < \frac{1}{n}$ и $\rho_y(f(x_n), b) \geqslant e$. \\
    
    Последовательность $x_n$ стремится к $a$, тогда последовательность $f(x_n)$ обязаана стремиться к $b$, но $\forall \, n \;\; \rho_y(f(x_n), b) \geqslant e$. Следовательно, мы пришли к противоречию.
\end{proof}

\begin{follow} 
    Предел единственен
\end{follow}
\begin{proof}
    Пусть $\lim\limits_{x \to a} f(x) = b$ и $\lim\limits_{x \to a} f(x) = c$, где $a$ -- предельная точка $E$. Возьмем последовательность $x_n \in E, \; x_n \neq a$ и $\lim x_n = a$. Тогда по определению по Гейне $\lim f(x_n) = b$ и $\lim f(x_n) = c \Rightarrow b = c$.  
\end{proof}