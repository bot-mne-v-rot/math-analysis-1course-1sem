\section{Определение и простейшие свойства площади и квазиплощади \href{https://youtu.be/p9C57KDo1Yg?t=6438}{\Walley}}

$\mathcal{F}$ - совокупность всех ограниченных подмножеств плоскости

\begin{conj}
    $\sigma: \mathcal{F} \to [0, +\infty)$ - площадь, если

    1) $\sigma([a, b]\times[c, d]) = (b-a)(c-d)$

    2) Если $E_1\cap E_2 = \varnothing$, то $\sigma(E_1 \cup E_2) = \sigma(E_1) \cup \sigma(E_2)$
\end{conj}

Свойство. Если $E \subset \widetilde{E}$, то
$\sigma(E) \leq \sigma(\widetilde{E})$

\begin{proof}
    $\widetilde{E} = E \cup (\widetilde{E} \backslash E) \Rightarrow
    \sigma(\widetilde{E}) = \sigma(E) + \sigma(\widetilde{E} \backslash E) \geq \sigma(E)$
\end{proof}

\begin{conj}
    $\sigma: \mathcal{F} \to [0, +\infty)$ - квазиплощадь, если

    1) $\sigma([a, b]\times[c, d]) = (b-a)(c-d)$

    2) Если $E \subset \widetilde{E}$, то $\sigma(E) \leq \sigma(\widetilde{E})$

    3) $\sigma(E) = \sigma(E_{-}) + \sigma(E_{+})$. $E_{-}$ и $E_{+}$ получаются
    разбиением $E$ горизонтальной или вертикальной прямой.
\end{conj}

Свойства
\begin{enumerate}
    \item Площадь подмножества любого вертикального или горизонтального отрезка ноль (из пункта 1 определения)
    \item $E_{-}$ и $E_{+}$ могут пересекаться 
\end{enumerate}

\begin{proof}
    Знаем, что $\sigma(E) = \sigma(E_-) + \sigma(E_+)$

    Каждую половину можно снова расщепить по той же прямой и показать, что эту полоску 
    можно добавлять или выкидывать сколько угодно раз.
    Более формально:

    $E_-' := \sigma(E_- \backslash l),\ 
    E_-'' = E_- \cup (E \cap l)$

    $E_-' \subset E_- \subset E_-''$

    \[\sigma(E_-') \leq \sigma(E_-) \leq \sigma(E_-'') = \sigma(E_-') \sigma(E \cup l) = \sigma(E_-') \Rightarrow
    \sigma(E_-) = \sigma(E_-') = \sigma(E_-'')\]
\end{proof}
