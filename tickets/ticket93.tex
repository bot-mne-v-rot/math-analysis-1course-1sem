\section{Определение и простейшие свойства площади и квазиплощади}

$\mathcal{F}$ - совокупность всех ограниченных подмножеств плоскости

\begin{conj}
    $\sigma: \mathcal{F} \to [0, +\infty)$ - площадь, если

    1) $\sigma([a, b]\times[c, d]) = (b-a)(c-d)$

    2) Если $E_1\cap E_2 = \varnothing$, то $\sigma(E_1 \cup E_2) = \sigma(E_1) \cup \sigma(E_2)$
\end{conj}

Свойство. Если $E \subset \widetilde{E}$, то
$\sigma(E) \leq \sigma(\widetilde{E})$

\begin{proof}
    $\widetilde{E} = E \cup (\widetilde{E} \backslash E) \Rightarrow
    \sigma(\widetilde{E}) = \sigma(E) + \sigma(\widetilde{E} \backslash E) \geq \sigma(E)$
\end{proof}

\begin{conj}
    $\sigma: \mathcal{F} \to [0, +\infty)$ - квазиплощадь, если

    1) $\sigma([a, b]\times[c, d]) = (b-a)(c-d)$

    2) Если $E \subset \widetilde{E}$, то $\sigma(E) \leq \sigma(\widetilde{E})$

    3) $\sigma(E) = \sigma(E_{-}) + \sigma(E_{+})$. $E_{-}$ и $E_{+}$ получаются
    разбиением $E$ горизонтальной или вертикальной прямой.
\end{conj}

Свойства
\begin{enumerate}
    \item Площадь подмножества любого вертикального или горизонтального отрезка ноль (из пункта 1 определения)
    \item $E_{-}$ и $E_{+}$ могут пересекаться (из прерыдущего свойства)
\end{enumerate}