\section{Пример квазиплощади, определенной на всех подмножествах плоскости}

Примеры квазиплощади

1)
\[\sigma_1(E) = \inf\{\sum_{k=1}^{n}|P_k| : E \subset \overset{n}{\underset{k=1}{\cup}}\} \]

$P_k$ - прямоугольники (конечное покрытие). $|P_k|$ - стандартная площадь прямоугольника.

2)
\[\sigma_1(E) = \inf\{\sum_{k=1}^{\infty}|P_k| : E \subset \overset{n}{\underset{k=1}{\cup}}\} \]

То же самое, только счетное покрытие

Замечание. $\sigma_1(E) \geq \sigma_2(E)$

Упражнение. $E = ([0, 1] \cap \Q)^2$. Доказать, что $\sigma_1(E) = 1,\ \sigma_2(E) = 0$

Упражнение. $\sigma_1,\ \sigma_2$ не меняются при параллельном переносе

\begin{theorem}
    $\sigma_1$ - квазиплощадь.
\end{theorem}

\begin{proof}
    Свойство 2. Рассмотрим покрытие $\widetilde{E} \subset \overset{n}{\underset{k=1}{\cup}} P_k
    \Rightarrow E \subset \overset{n}{\underset{k=1}{\cup}}$

    Тогда инфимум для $E$ берется по большему множеству, значит он может быть только меньше

    Свойство 3. $"\leq":$
    \[E_{-} \subset \overset{n}{\underset{k=1}{\cup}} P_k,\ 
    E_{+} \subset \overset{m}{\underset{j=1}{\cup}} Q_j\]

    \[E \subset \overset{n}{\underset{k=1}{\cup}} P_k\ \cup\ \overset{m}{\underset{j=1}{\cup}} Q_j\]

    \[\sigma_1(E) \leq \sum |P_k| + \sum |Q_j| \Rightarrow \sigma(E) \leq \inf
    \{ \sum |P_k| \} + \sum |Q_j| = \sigma_1(E_{-}) + \sum |Q_j| \Rightarrow \]

    \[ \Rightarrow \sigma_1(E) \leq \sigma_1(E_{-}) + \inf \{ \sum |Q_k| \} =
    \sigma_1(E_{-}) + \sigma_1 + (E_{+}) \]

    $"\geq"$

    \[ \sigma_1(E_{-}) + \sigma_1(E_{+}) \leq \sum |P_k^{-}| + \sum |P_k^{+}| = \sum |P_k|\]

    Переходим к $\inf$

    \[ \sigma_1(E_{-}) + \sigma_1{E_{+}} \leq \sigma_1(E) \]

    Свойство 1.

    $"\leq"$ очевидно, так как в качестве покрытия можно взять сам прямоугольник

    $"\geq"$. Если прямугольник уже идеально нарезан, то получилось равенство.
    Для произвольного покрытия можем отрезать лишние куски, выкинуть повторяющиеся куски и получить идеальную нарезку.
\end{proof}