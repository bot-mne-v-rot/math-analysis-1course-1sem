\section{Пример квазиплощади, определенной на всех подмножествах плоскости \href{https://youtu.be/p9C57KDo1Yg?t=7344}{\Walley}}

Примеры квазиплощади

\begin{enumerate}
    \item \[\sigma_1(E) = \inf \bigg\{ \sum_{k=1}^{n}|P_k| : E \subset \bigcup\limits_{k=1}^n \bigg\} \]

$P_k$ - прямоугольники (конечное покрытие). $|P_k|$ - стандартная площадь прямоугольника.
    \item \[\sigma_1(E) = \inf \bigg\{ \sum_{k=1}^{\infty}|P_k| : E \subset \bigcup\limits_{k=1}^n \bigg\} \]

То же самое, только счетное покрытие
\end{enumerate}

\notice \; $\sigma_1(E) \geqslant \sigma_2(E)$

\textbf{Упражнение:} $E = ([0, 1] \cap \Q)^2$. Доказать, что $\sigma_1(E) = 1,\ \sigma_2(E) = 0$

\textbf{Упражнение:} $\sigma_1,\ \sigma_2$ не меняются при параллельном переносе

\begin{theorem-non}
    $\sigma_1$ - квазиплощадь.
\end{theorem-non}

\begin{proof}
    Свойство 2. Рассмотрим покрытие $\widetilde{E} \subset \bigcup\limits_{k=1}^n P_k
    \Rightarrow E \subset \bigcup\limits_{k=1}^n$

    Тогда инфимум для $E$ берется по большему множеству, значит он может быть только меньше

    Свойство 3. $"\leqslant":$
    \[E_{-} \subset \bigcup\limits_{k=1}^n P_k,\ 
    E_{+} \subset \bigcup\limits_{j=1}^m Q_j\]

    \[E \subset \bigcup\limits_{k=1}^n P_k\ \cup\ \bigcup\limits_{j=1}^m Q_j\]

    \[\sigma_1(E) \leqslant \sum |P_k| + \sum |Q_j| \Rightarrow \sigma(E) \leqslant \inf
    \Big\{ \sum |P_k| \Big\} + \sum |Q_j| = \sigma_1(E_{-}) + \sum |Q_j| \Rightarrow \]

    \[ \Rightarrow \sigma_1(E) \leqslant \sigma_1(E_{-}) + \inf \Big\{ \sum |Q_k| \Big\} =
    \sigma_1(E_{-}) + \sigma_1 + (E_{+}) \]

    $"\geqslant"$

    \[ \sigma_1(E_{-}) + \sigma_1(E_{+}) \leqslant \sum |P_k^{-}| + \sum |P_k^{+}| = \sum |P_k|\]

    Переходим к $\inf$

    \[ \sigma_1(E_{-}) + \sigma_1{E_{+}} \leqslant \sigma_1(E) \]

    Свойство 1.

    $"\leqslant"$ очевидно, так как в качестве покрытия можно взять сам прямоугольник

    $"\geqslant"$. Если прямугольник уже идеально нарезан, то получилось равенство.
    Для произвольного покрытия можем отрезать лишние куски, выкинуть повторяющиеся куски и получить идеальную нарезку.
\end{proof}