\section{Формула Тейлора для многочленов}

\begin{lemma}

    Пусть $f(x) = (x -x_0)^k$, тогда $f^{(m)}(x_0) =
    \begin{cases}
        m!, &\text{если $k=m$}\\
        0, &\text{иначе}
    \end{cases}$
\end{lemma}

\begin{proof}  
    $\quad$ \\
    \begin{itemize}
        \item $k \geqslant m$, тогда $f^{(m)}(x) = k(k-1) \dots (k-m+1)(x-x_0)^{k-m}$
        \begin{itemize}
            \item если $k = m$, то $f^{(m)}(x) = k! = m!$
            \item если $k > m$, то $f^{(m)}(x) = c (x-x_0)^{k-m} \Longrightarrow f^{(m)}(x_0) = 0$
        \end{itemize}
        
        \item $k < m$, тогда $(f^{(k)})^{(m-k)} = (k!)^{(m-k)} = 0$
    \end{itemize}
\end{proof}

\begin{theorem-non}
    Формула Тейлора для многочлена

    $T$ --- многочлен, $deg \, T \leqslant n$. 

    Тогда $T(x) = \sum\limits_{k=0}^n \frac{T^{(k)}(x_0)}{k!} (x-x_0)^k$.
\end{theorem-non}
\begin{proof} $\quad$ \\
    Разложим $T$ по степеням $(x - x_0)$: $T(x) = \sum\limits_{k = 0}^{n} c_k (x-x_0)^k$ (умеем из алгебры)
    
    Надо доказать, что $c_k = \frac{T^{(k)}(x_0)}{k!}$.

    $T^{(m)}(x_0) = \left.\sum\limits_{k = 0}^{n} c_k ((x-x_0)^k)^{(m)} \right|_{x=x_0} = c_m \cdot m!$ (по лемме)

    Тогда $c_m = \frac{T^{(m)}(x_0)}{m!}$.
\end{proof}
