\section{Существование предела в терминах последовательностей. Локальная ограниченность функции, имеющей предел. Критерий Коши \href{https://youtu.be/BiTNBigkkyU?t=2205}{\Walley}}

\begin{theorem-non}
    Если в определении по Гейне все пределы существуют, то они равны между собой. 
\end{theorem-non}
\begin{proof}
    Рассмотрим две последовательности $x_n \in E, \; x_n \neq a$ и $y_n \in E, \; y_n \neq a$ такие, что $\lim x_n = \lim y_n = a$. Покажем, что $\lim f(x_n) = \lim f(y_n)$
    
    Положим $b := \lim f(x_n)$ и $c := \lim f(y_n)$.

    Возьмем последовательность $x_1, y_1, x_2, y_2, \dots := z_n$. Тогда $\lim z_n = a \Rightarrow \exists \, \lim f(z_n) =: d$.
    
    $f(x_n)$ -- подпоследовательность $f(z_n) \Rightarrow b = d$ \\
    $f(y_n)$ -- подпоследовательность $f(z_n) \Rightarrow c = d$ \\
    $\Rightarrow b = c$
\end{proof}

\begin{theorem-non}
    $f : E \to Y, \; E \subset X, \; a$ -- предельная точка $E$ и $\lim\limits_{x \to a} f(x) = b$. 

    Тогда для некоторого $r > 0$ $f$ ограничена на $B_r(a) \cap E$. 
\end{theorem-non}
\begin{proof}
    Зафиксируем $\varepsilon = 1$ в определении по Коши. Найдем $\delta > 0$ такое, что если $x \in E$ и $0 < \rho_x(x, a) < \delta$, то $\rho_y(f(x), b) < 1$.

    Рассмотрим $f$ на $B_{\delta}(a) \cap E$:
    \[ \rho_y(f(x), b) \leqslant max \{1, \rho_y(f(a), b)\} \]
    $\Rightarrow f$ ограничена на $B_{\delta}(a) \cap E$.
\end{proof}
\begin{notice}
    Глобальной ограниченности нет. 

    Рассмотрим $f(x) = \frac{1}{x}$. Тогда $\lim\limits_{x \to 1} f(x) = 1$, но $f$ очевидно не огранечена.
\end{notice}

\vspace{7mm}

\begin{theorem-non}
    Критерий Коши для отображений

$f : E \to Y, \; E \subset X, \; a$ -- предельная точка $E$, и $Y$ -- полное. Тогда
\[ \exists \, \lim_{x \to a} f(x) \Longleftrightarrow \forall \, \varepsilon > 0 \;\; \exists \, \delta > 0 \;\; \forall \, x, \, y \in (\overset{\circ}{B}_{\delta}(a) \cap E) \Rightarrow \rho_y(f(x), f(y)) < \varepsilon \]
\end{theorem-non}

\begin{proof} \quad \\
    "$\Rightarrow$": Пусть $\lim\limits_{x \to a} f(x) = b$. Тогда из определения по Коши:
    \begin{gather*}
        \forall \, \varepsilon > 0 \;\; \exists \, \delta > 0 \;\; \forall \, x \in (\overset{\circ}{B}_{\delta}(a) \cap E) \Rightarrow \rho_y(f(x), b) < \frac{\varepsilon}{2} \\
        \forall \, y \in (\overset{\circ}{B}_{\delta}(a) \cap E) \Rightarrow \rho_y(f(y), b) < \frac{\varepsilon}{2} \\
        \Rightarrow \rho_y(f(x), f(x)) < \varepsilon
    \end{gather*}
    "$\Leftarrow$": Возьмем последовательность $x_n \in E, \; x_n \neq a$ такую, что $\lim x_n = a$. Хотим доказать, что $\lim f(x_n)$ существует. Покажем, что $f(x_n)$ -- фундаментальная последовательность.

    Возьмем $\varepsilon > 0$. По нему $\delta > 0$ из критерия Коши, а по $\delta$ найдем $N$ из предела $\lim x_n = a$ (т.е. номер, начиная с которого $x_n$ будет лежать в $\delta$-окрестности $a$). Формально:
    \begin{gather*}
        \exists \, N \;\; \forall \, n \geqslant N \;\; \rho_x(x_n, a) < \delta \\
        \forall \, m \geqslant N \;\; \rho_x(x_m, a) < \delta \\
    \end{gather*}
    Подставив $x_n$ и $x_m$ в критерий Коши, получаем, что $\rho_y(f(x_n), f(x_m)) < \varepsilon$
    $\Rightarrow$ $f(x_n)$ -- фундаментальная $\Rightarrow \exists \, \lim f(x_n)$.
\end{proof}

\begin{theorem-non}
    Критерий Коши для функций

$f : E \to \R, \; E \subset \R, \, a$ -- предельная точка $E$
\[ 
\exists \, \lim_{x \to a} f(x) \Longleftrightarrow \forall \, \varepsilon > 0 \;\; \exists \,\delta > 0 : 
\begin{cases} 
    \forall \, x \in E : \, 0 < |x - a| < \delta \\
    \forall \, y \in E : \, 0 < |y - a| < \delta \\
\end{cases}    
    \Rightarrow |f(x) - f(y)| < \varepsilon
\]
\end{theorem-non}
\begin{notice}
    Данный предел будет конечным. Для отображений это очевидно, там бесконечного предела нет по определению.
\end{notice}

\begin{conj}
    Бесконечные пределы для функций
\end{conj}
Определение по Гейне никак не меняется. 

Опредление по Коши:
\[ \lim_{x \to a} f(x) = +\infty \Longleftrightarrow \forall \, u \;\; \exists \, \delta > 0 : \; \forall \, x \in E : 0 < |x - a| < \delta \Rightarrow f(x) > u \]
\[ \lim_{x \to a} f(x) = -\infty \Longleftrightarrow \forall \, u \;\; \exists \, \delta > 0 : \; \forall \, x \in E : 0 < |x - a| < \delta \Rightarrow f(x) < u \]

\begin{notice}
    Также можно определить пределы в $+\infty$ и $-\infty$. Например:
    \[ \lim_{x \to +\infty} f(x) = b \Longleftrightarrow \forall \, \varepsilon > 0 \;\; \exists \, u : \; \forall \, x \in E : x > u \Rightarrow |f(x) - b| < \varepsilon   \]
    \[ \lim_{x \to -\infty} f(x) = +\infty \Longleftrightarrow \forall \, u \;\; \exists \, v : \; \forall \, x \in E : x < v \Rightarrow f(x) > u \]
\end{notice}