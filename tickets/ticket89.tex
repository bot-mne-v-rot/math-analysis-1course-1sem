\section{Неравенства Гёльдера, Коши–Буняковского и Минковского}

\begin{theorem-non}
    Неравенство Гёльдера. 
    $a_k, b_k \geqslant 0,\ p, q > 1,\ \frac{1}{p}+\frac{1}{q} = 1$. Тогда
    
    \[ \sum_{k=1}^{n} a_k b_k \leqslant \left( \sum_{k=1}^{n} a_k^p\right)^\frac{1}{p} \left( \sum_{k=1}^{n} b_k^q\right)^\frac{1}{q} \]
\end{theorem-non}

\begin{proof}
    При умножении всех $b$ на какую-то константу неравенство остается верным. Тогда можно считать, что $b_1^q+b_2^q+\cdots+b_n^q=1$

    Это получается для каждой $b$ делением на $\sqrt[q]{b_1^q+b_2^q+\cdots+b_n^q}$

    Тогда нужно доказать

    \[ \left(\sum a_k b_k \right)^p \leqslant \sum a_k^p \]

    Возьмем $f(x) = x^p$, она выпуклая. Хотим привести к виду

    \[ f(\sum \lambda_k x_k) \leqslant \sum \lambda_k f(x_k) \]

    $
    \begin{cases}
        \lambda_k x_k = a_k b_k \\
        \lambda_k x_k^p = a_k^p
    \end{cases}
    $

    Поделим вторую строчку на первую, тогда

    \[ x_k^{p-1} = \frac{a_k^{p-1}}{b_k} \Rightarrow x_k = \frac{a_k}{b_k^{\frac{1}{p-1}}},\ 
    \lambda_k = \frac{a_k b_k}{x_k} = b^{1+\frac{1}{p-1}} = b_k^q\]

    Мы свели к неравенству Йенсена

    $\frac{1}{p}+\frac{1}{q} = 1$ используется в последнем переходе, и без него было бы неверно, что сумма лямбд равна $1$

    \begin{notice}
        Если $\frac{1}{p}+\frac{1}{q} = 1$ и $0 < p < 1$, то $q < 0$ и верно неравенство с обратным знаком.
    \end{notice}
\end{proof}

\begin{theorem-non}
    Неравенство Коши–Буняковского.

    \[ (\sum_{k=1}^{n} a_k b_k)^2 \leqslant \sum_{k=1}^{n} a_k^2 \cdot \sum_{k=1}^{n}b_k^2 \]
\end{theorem-non}

\begin{proof}
    $p = q = 2$, тогда по неравенству Гёльдера
    \[ |\sum a_kb_k| \leqslant \sum|a_kb_k| \leqslant \left(\sum|a_k|^2\right)^\frac{1}{2} \left(\sum|b_k|^2\right)^\frac{1}{2} =
    \left(\sum a_k^2 \sum b_k^2\right)^\frac{1}{2}\]

    Возводим самую левую и самую правую часть в квадрат и получаем искомое неравенство.
\end{proof} 

\begin{theorem-non}
    Неравенство Минковского.

    $a_k, b_k \geqslant 0,\ p \geqslant 1$. Тогда

    \[ \left(\sum_{k=1}^{n}(a_k+b_k)^p\right)^\frac{1}{p} \leqslant \left(\sum_{k=1}^{n} a_k^p\right)^\frac{1}{p} +
    \left(\sum_{k=1}^{n} b_k^p\right)^\frac{1}{p} \]
\end{theorem-non}

\begin{proof}
    $p = 1$ очевидно (написано тождество). Пусть $p > 1$.

    \[ \sum_{k=1}^{n}(a_k+b_k)^p = \sum_{k=1}^{n}(a_k+b_k)(a_k+b_k)^{p-1} =
    \sum_{k=1}^{n} a_k(a_k+b_k)^{p-1} + \sum_{k=1}^{n} b_k(a_k+b_k)^{p-1} \]

    Подберем под это неравенство Гёльдера:

    $q = \frac{p}{p-1}$ (тогда $\frac{1}{p} + \frac{1}{q} = 1$)

    \[ \sum_{k=1}^{n} a_k(a_k+b_k)^{p-1} \leqslant \left( \sum a_k^p \right)^\frac{1}{p} \left( \sum ((a_k+b_k)^{p-1})^q \right)^\frac{1}{q}
    = \left( \sum a_k^p \right)^\frac{1}{p} \left( \sum (a_k+b_k)^p \right)^\frac{1}{q} \] 

    Аналогично для второго слагаемого. Тогда продолжим нашу цепочку неравенств:

    \[ \sum_{k=1}^{n} a_k(a_k+b_k)^{p-1} + \sum_{k=1}^{n} b_k(a_k+b_k)^{p-1} \leqslant
    \left( \sum a_k^p \right)^\frac{1}{p} \left( \sum (a_k+b_k)^p \right)^\frac{1}{q} + \left( \sum b_k^p \right)^\frac{1}{p} \left( \sum (a_k+b_k)^p \right)^\frac{1}{q} \]

    Делим самую левую и самую правую часть на $\left( \sum (a_k+b_k)^p \right)^\frac{1}{q}$, 
    получаем то что надо (при вычислении степеней не забывать, что $1/p + 1/q = 1$)


\end{proof}

\begin{theorem-non}
    Следствие: $x_k, y_k, z_k \in \R,\ p \geqslant 1$. Тогда

    \[ \left( \sum |x_k-y_k|^p \right)^\frac{1}{p} + \left( \sum |y_k-z_k|^p \right)^\frac{1}{p} \geqslant
    \left( \sum |z_k-x_k|^p \right)^\frac{1}{p} \]
    
\end{theorem-non}

\begin{proof}
    $a_k = |x_k - y_k|,\ b_k = |y_k - z_k|$

    \[ \left( \sum a^p \right)^\frac{1}{p} + \left( \sum b^p \right)^\frac{1}{p} 
    \geqslant \left( \sum (a_k+b_k)^p \right)^\frac{1}{p} \geqslant \left( \sum |z_k-x_k|^p \right)^\frac{1}{p} \]
    
\end{proof}

\begin{notice}
$\rho(x, y) = \left( \sum |x_k-y_k|^p \right)^\frac{1}{p}$ - метрика

$||x|| := \left( \sum |x_k|^p \right)^\frac{1}{p}$ - норма
\end{notice}