\section{Неравенство Йенсена, неравенство о средних}

\begin{theorem-non}
    Неравенство Йенсена.
    
    $\lambda_1, \cdots, \lambda_n \geqslant 0$, 
    $\lambda_1+\cdots+\lambda_n = 1$

    $f: \langle a, b \rangle \to \R$ выпуклая

    $x_1,\cdots, x_n \in \langle a, b \rangle$

    Тогда \[f(\lambda_1x_1+\cdots+\lambda_nx_n)\leqslant \lambda_1f(x_1)+
    \cdots+\lambda_nf(x_n) \]
\end{theorem-non}

\begin{proof}
    Индукция. База $n = 2$ - определение выпуклости. Переход $n \to n+1$

    $ (1-\lambda_{n+1}) y:= \lambda_1x_1+\cdots+\lambda_n x_n$

    \[ f(\lambda_1x_1+\cdots+\lambda_n x_n + \lambda_{n+1}x_{n+1}) = f(\lambda_{n+1}x_{n+1} + (1-\lambda_{n+1})y)
    \leqslant \lambda_{n+1}f(x_{n+1}) + (1-\lambda_{n+1})f(y) = \]

    \[ \lambda_{n+1}f(x_{n+1}) + (1-\lambda_{n+1}) f \left( \frac{\lambda_1}{1-\lambda_{n+1}}x_1 + \frac{\lambda_2}{1-\lambda_{n+1}}x_2+
    \cdots + \frac{\lambda_n}{1-\lambda_{n+1}}x_n \right) \leqslant \]

    Применим индукционное предположение. Можем так сделать, потому что $\sum\limits_{k = 1}^n \frac{\lambda_k}{1 - \lambda_{n + 1}} = 1$.

    \[ \leqslant \lambda_{n+1}f(x_{n+1}) + (1-\lambda_{n+1}) \left( \frac{\lambda_1}{1-\lambda_{n+1}}f(x_1) + \frac{\lambda_2}{1-\lambda_{n+1}}f(x_2)+
    \cdots + \frac{\lambda_n}{1-\lambda_{n+1}}f(x_n) \right) = \]

    \[ = \lambda_1f(x_1)+ \cdots+\lambda_nf(x_n) \]

    Осталось показать, что $a \leq y \leq b$ (там может быть отрезок или интервал, но на суть это не влияет)

    Оценим иксы снизу и сверху как $a$ и $b$ соответственно

    \[ \lambda_1 a+\cdots+\lambda_n a \leq (1-\lambda_{n+1})\cdot y \leq \lambda_1 b+\cdots+\lambda_n b \]

    Делим и получаем $a \leqslant y \leqslant b$

    Графический смысл теоремы - центр масс лежит над графиком
    
\end{proof}

\begin{theorem-non}
    Неравенство о средних (AM-GM - arigthmetic mean, geometry mean)

    $x_1, x_2, \cdots, x_n \geqslant 0$. Тогда

    $\sqrt[n]{x_1x_2\cdots x_n} \leqslant \frac{x_1+x_2+\dots+x_n}{n}$
\end{theorem-non}

\begin{proof}
    Докажем, что \[\ln (\sqrt[n]{x_1x_2\dots x_n}) = \frac{1}{n}(\ln x_1 + \ln x_2 + \dots + \ln x_n) 
    \leqslant \ln (\frac{x_1+x_2+\dots+x_n}{n})\]

    Будем считать, что все $x > 0$, так как если какой-то равен 0, то неравенство очевидно

    
    Это неравенство Йенсена для $f(x) = -\ln x$ - выпуклая.
    $\lambda_1 = \lambda_2 = \cdots = \lambda_n = \frac{1}{n} $.
\end{proof}