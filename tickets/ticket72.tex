\section{Следствия теоремы Лагранжа. Характеристика монотонности дифференцируемых функций}
\textbf{Следствия из т. Лагранжа:}
\begin{enumerate}
    \item Пусть
    \begin{itemize}
        \item $f$ непрерывна на $\langle a, b \rangle$ и дифференцируема на $(a, b)$
        \item $|f'(x)| \leqslant M$ при всех $x \in (a, b)$
    \end{itemize}
    Тогда $|f(x) - f(y)| \leqslant M|x - y| \quad \forall x, y \in \langle a, b \rangle$
    \begin{proof}
        Будем считать, что $x < y$. Посмотрим на отрезок $[x, y]$. $f$ непрерывна на $[x, y]$ и дифференцируема на $(x, y)$. 

        $\Rightarrow \exists \, c \in (x, y) : f(y) - f(x) = f'(c)(y - x)$

        $\Rightarrow |f(y) - f(x)| = |f'(c)||y - x| \leqslant M|y - x|$
    \end{proof}
    \begin{notice}
        Такая функция $f$ называется липшицевой с константой $M$.
    \end{notice}
    \item Пусть
    \begin{itemize}
        \item $f$ непрерывна на $\langle a, b \rangle$ и дифференцируема на $(a, b)$
        \item $f'(x) \geqslant 0$ при всех $x \in (a, b)$
    \end{itemize}
    Тогда $f$ нестрого монотонно возрастает.
    \begin{proof}
        Возьмем $x, \, y \in \langle a, b \rangle$, причем $x < y$. 
        Посмотрим на отрезок $[x, y]$. $f$ непрерывна на $[x, y]$ и дифференцируема на $(x, y)$. 

        $\Rightarrow \exists \, c \in (x, y) : f(y) - f(x) = \underbrace{f'(c)}_{\geqslant 0}\underbrace{(y - x)}_{> 0} \Rightarrow f(y) \geqslant f(x)$
    \end{proof}
    \item Если $f'(x) > 0$ при всех $x \in (a, b)$, то $f$ строго монотонно возрастает.
    \item Если $f'(x) \leqslant 0$ при всех $x \in (a, b)$, то $f$ нестрого монотонно убывает.
    \item Если $f'(x) < 0$ при всех $x \in (a, b)$, то $f$ строго монотонно убывает.
    \item Если $f'(x) = 0$ при всех $x \in (a, b)$, то $f$ постоянна.
\end{enumerate}

\begin{theorem-non}
    Пусть $f: \langle a, b \rangle \to \mathbb{R}$ непрерывна на $\langle a, b \rangle$ и дифференцируема на $(a, b)$.

    Тогда
    \begin{enumerate}
        \item $f$ нестрого монотонно возрастает $\Leftrightarrow$ $f'(x) \geqslant 0 \;\; \forall x \in (a, b)$
        \item $f$ нестрого монотонно убывает $\Leftrightarrow$ $f'(x) \leqslant 0 \;\; \forall x \in (a, b)$
    \end{enumerate}
\end{theorem-non}
\begin{proof}
    Докажем первый пункт, второй аналогично.

    $\quad \Leftarrow:$ доказано выше

    $\quad \Rightarrow:$ 
    \[ f'(x) = f_+'(x) = \lim_{y \to x+} \frac{f(y) - f(x)}{y - x} = \lim \frac{\geqslant 0}{> 0} = \lim (\geqslant 0) \geqslant 0 \]
\end{proof}
\begin{notice}
    $f$ строго возрастает $\nRightarrow f' > 0$. Например: $f(x) = x^3, \; f'(0) = 0$
\end{notice}