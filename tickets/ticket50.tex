\section{Левый и правый пределы. Предел монотонной функции}

\vspace{7mm}

\begin{conj}
    Односторонние пределы
\end{conj}
$f : E \to Y, \, E \subset \R$
\begin{itemize}
    \item $a$ -- предельная точка множества $E \cap (a, +\infty) =: E_1$
    
    Сузим $f$ на $E_1$. Назовем получившуюся функцию $f_1$. 
    Если у $f_1$ есть предел в точке $a$, то говорят, что у $f$ есть правосторонний предел:
    \[ \lim_{x \to a+} f(x) = b \Longleftrightarrow \lim_{x \to a} f_1(x) = b \Longleftrightarrow 
    \forall \, \varepsilon > 0 \;\; \exists \, \delta > 0 : \; \forall \, x \in E : 0 < x - a < \delta \Rightarrow\rho_y(f(x), b) < \varepsilon \]
    
    \item $a$ -- предельная точка множества $E \cap (-\infty, a) =: E_2$
    
    Сузим $f$ на $E_2$. Назовем получившуюся функцию $f_2$.
    Если у $f_2$ есть предел в точке $a$, то говорят, что у $f$ есть левосторонний предел:
    \[ \lim_{x \to a-} f(x) = c \Longleftrightarrow \lim_{x \to a} f_2(x) = c \Longleftrightarrow 
    \forall \, \varepsilon > 0 \;\; \exists \, \delta > 0 : \; \forall \, x \in E : 0 < a - x < \delta \Rightarrow\rho_y(f(x), c) < \varepsilon \]
\end{itemize}
\begin{conj}
    Односторонние пределы по Гейне
\end{conj}
\begin{itemize}
    \item $\lim\limits_{x \to a+} f(x) = b \Longleftrightarrow$ для любой последовательности $x_n \in E, \, x_n > a$ и $\lim x_n = a$ имеем $\lim f(x_n) = b$
    \item $\lim\limits_{x \to a-} f(x) = c \Longleftrightarrow$ для любой последовательности $x_n \in E, \, x_n < a$ и $\lim x_n = a$ имеем $\lim f(x_n) = c$
\end{itemize}
\underline{Пример:} $f(x) = [x], \, n \in \N$

$\lim\limits_{x \to n+} f(x) = n$, но $\lim\limits_{x \to n-} f(x) = n - 1$

\begin{notice}
    $\lim\limits_{x \to a} f(x) = b \Longleftrightarrow \lim\limits_{x \to a+} f(x) = \lim\limits_{x \to a-} f(x)$
\end{notice}

\vspace{7mm}

\begin{theorem-non}
    Существование предела у монотонной ограниченной функции

    $f : E \to \R, \, E \subset \R, \, a$ -- предельная точка множества $E \cap (-\infty, a)$

    Если $f$ возрастает (убывает) и ограничена сверху (снизу), то существует конечный $\lim\limits_{x \to a-} f(x)$.
\end{theorem-non}
\begin{proof}
    Пусть $f$ возрастает и огранечена сверху $\Rightarrow$ существует конечный $\sup f(x) =: b$, где $x \in E \cap (-\infty, a)$.
    Докажем, что $\lim\limits_{x \to a-} f(x) = b$.

    Возьмем $\varepsilon > 0$. Мы знаем, что $b - \varepsilon$ не является верхней границей, следовательно найдется $y \in E \cap (-\infty, a)$ такой, что $f(y) > b - \varepsilon$.


    Если $y < x < a$, то $\begin{cases} f(x) \geqslant f(y) > b - \varepsilon \\ f(x) \leqslant b < b + \varepsilon \end{cases} \Rightarrow |f(x) - b| < \varepsilon \Rightarrow \lim\limits_{x \to a-} f(x) = b$
\end{proof}