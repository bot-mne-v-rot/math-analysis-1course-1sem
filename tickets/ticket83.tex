\section{Достаточные условия экстремума для дифференцируемых функций}
\begin{theorem-non}
    Достаточные условия экстремума в терминах первой производной.

    $f: \langle a, b \rangle: \to \mathbb{R}, \, x_0 \in (a, b)$ \\
    $f$ непрерывна в точке $x_0$ и дифференцируема на $(x_0 - \delta, x_0)$ и $(x_0, x_0 + \delta)$ для какого-то $\delta$
    \begin{enumerate}
        \item Если $f'(x) < 0$ на $(x_0 - \delta, x_0)$ и $f'(x) > 0$ на $(x_0, x_0 + \delta)$, то $x_0$ -- точка строгого локального минимума.
        \item Если $f'(x) > 0$ на $(x_0 - \delta, x_0)$ и $f'(x) < 0$ на $(x_0, x_0 + \delta)$, то $x_0$ -- точка строгого локального максимума.
        \item Если $f'$ не меняет знак в точке $x_0$, то $x_0$ не является точкой экстремума. 
    \end{enumerate}
\end{theorem-non}

\begin{proof}
    Докажем первый пункт. Второй и третий доказываются аналогично.
    \begin{itemize}
        \item $f$ непрерывна на $(x_0 - \delta, x_0]$, дифференцируема на $(x_0 - \delta, x_0)$ и $f' < 0 \Rightarrow f$ строго убывает на $(x_0 - \delta, x_0] \Rightarrow f(x) > f(x_0)$ при $x \in (x_0 - \delta, x_0)$
        \item  $f$ непрерывна на $[x_0, x_0 + \delta)$, дифференцируема на $(x_0, x_0 + \delta)$ и $f' > 0 \Rightarrow f$ строго возрастает на $[x_0, x_0 + \delta) \Rightarrow f(x_0) < f(x)$ при $x \in (x_0, x_0 + \delta)$
    \end{itemize}
    $\quad \Rightarrow x_0$ -- точка локального минимума
\end{proof}

\begin{theorem-non}
    Достаточные условия экстремума в терминах второй производной.

    $f: \langle a, b \rangle: \to \mathbb{R}, \, x_0 \in (a, b)$ \\
    $f$ дважды дифференцируема в точке $x_0$ (т.е $f$ дифференцируема в окрестности $x_0$) и $f'(x_0) = 0$
    \begin{enumerate}
        \item Если $f''(x_0) > 0$, то $x_0$ -- точка строгого локального минимума.
        \item Если $f''(x_0) < 0$, то $x_0$ -- точка строгого локального максимума.
        \item Если $f''(x_0) = 0$, то ничего однозначно не утверждается.
    \end{enumerate}

    Это частный случай следующей теоремы.
\end{theorem-non}

\begin{theorem-non}
    Достаточные условия экстремума в терминах $n$-ой производной.

    $f: \langle a, b \rangle: \to \mathbb{R}, \, x_0 \in (a, b)$ \\
    $f \; n$ раз дифференцируема в точке $x_0$ и $f'(x_0) = f''(x_0) = \dots = f^{(n - 1)}(x_0) = 0$, но $f^{(n)}(x_0) \neq 0$
    \begin{enumerate}
        \item Если $n$ четно и $f^{(n)}(x_0) > 0$, то $x_0$ -- точка строгого локального минимума.
        \item Если $n$ четно и $f^{(n)}(x_0) < 0$, то $x_0$ -- точка строгого локального максимума.
        \item Если $n$ нечетно, то $x_0$ не является точкой экстремума. 
    \end{enumerate}
\end{theorem-non}

\begin{proof}
    \begin{enumerate} $\quad$ \\
        \item Запишем формулу Тейлора с остатком в форме Пеано для $f$: 
        \[ f(x) = \sum_{k = 0}^n \frac{f^{(k)}(x_0)}{k!} (x - x_0)^k + o((x - x_0)^n) \]
        Так как $f'(x_0) = f''(x_0) = \dots = f^{(n - 1)}(x_0) = 0$, от суммы останутся только слагаемые при $k = 0$ и $k = n$: 
        \[ f(x) = f(x_0) + \frac{f^{(n)}(x_0)}{n!} (x - x_0)^n + o((x - x_0)^n) \]
        Перекинем $f(x_0)$ в левую часть и вынесем $(x - x_0)^n$: \[ f(x) - f(x_0) = (x - x_0)^n(\frac{f^{(n)}(x_0)}{n!} + o(1)) \]
        $\frac{f^{(n)}(x_0)}{n!} + o(1) > 0$ в некоторой окрестности $x_0$, так как $f^{(n)}(x_0) > 0$. 
        При четном $n$ выражение $(x - x_0)^n$ всегда больше 0. 
        
        Получаем, что $f(x) - f(x_0) > 0$ при $x \to x_0$. Следовательно, $x_0$ -- точка строгого локального минимума.
        \item Аналогично первому пункту.
        \item При нечетном $n$ получаем, что $(x - x_0)^n(\frac{f^{(n)}(x_0)}{n!} + o(1))$ будет разных знаков с разных сторон от $x_0$. 
        Значит, $x_0$ не может быть экстремумом. 
    \end{enumerate}
\end{proof}
