% !TEX root = ../MatanColloc02.tex

\section{Равномерная непрерывность отображений. Примеры. Теорема Кантора для отображений метрических пространств.}


\begin{conj}
    Пусть $f : E \rightarrow Y$, $E \subset X$. \\
    $f$ \textbf{равномерно непрерывна} на $E$, если $\forall \varepsilon
    > 0 \quad \exists \delta > 0 : \forall x, y \in E : \rho_X(x, y) <
    \varepsilon \Rightarrow \rho_Y(f(x), f(y)) < \delta$.
\end{conj}

\textbf{Примеры:} $X = Y = \R$.
\begin{enumerate}
    \item $f(x) = x$ равномерно непрерывна.
    \item $f(x) = \sin x$ равномерно непрерывна, т.к.
    $\abs{\sin x - \sin y} \leqslant \abs{x - y}$.
    \item $f(x) = x^2$ не является равномерно непрерывной, хотя
    непрерывна во всех точках.
    \item $f(x) = \frac{1}{x}$ не является равномерно непрерывной.
\end{enumerate}

\begin{theorem-non}
    Кантора.
\end{theorem-non}
Пусть $f : K \rightarrow Y$ непрерывна, $K$ -- компакт. Тогда $f$ 
равномерно непрерывна.

\begin{proof} $ $

    Зафиксируем $\varepsilon > 0$. Возьмём $y \in K$ и найдём $r_y > 0$,
    т.ч. $f(B_{r_y}(y)) \subset B_{\varepsilon / 2}(f(y))$ по 
    определению непрерывности. Тогда $K \subset \bigcup \limits_{y \in K}
    B_{r_y}(y)$ -- покрытие открытыми множествами.

    Пусть $\delta > 0$ -- число Лебега для этого покрытия, т.е.
    $\forall x \in K \quad B_{\delta}(x)$ целиком попадает в какой-то
    элемент этого покрытия.

    Проверим, что это $\delta$ подходит, т.е. если $\rho(x, y) <
    \delta$, то $\rho(f(x), f(y)) < \varepsilon$.

    $\rho(x, y) < \delta \Rightarrow y \in f(x) \subset B_{r_a}(a)$
    для некоторого $a \in K$ $\Rightarrow x, y \in B_{r_a}(a)
    \Rightarrow f(x), f(y) \in f(B_{r_a}(a)) \subset B_{\varepsilon / 2}
    (f(a)) \Rightarrow
    \begin{cases}
        \rho(f(x), f(a)) < \varepsilon / 2 \\
        \rho(f(y), f(a)) < \varepsilon / 2
    \end{cases}
    \Rightarrow \rho(x, y) \leqslant \rho(f(x), f(a)) + \rho(f(y), f(a))
    < \varepsilon$.
\end{proof}
